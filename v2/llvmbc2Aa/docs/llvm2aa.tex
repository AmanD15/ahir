\documentclass{article}

\title{About {\bf llvm2aa}}
\author{Madhav Desai \\ Department of Electrical Engineering \\ Indian Institute of Technology \\
	Mumbai 400076 India}

\newcommand{\Aa}{{\bf Aa}~}
\newcommand{\vC}{{\bf vC}~}

\begin{document}
\maketitle

\section{Introduction}

{\bf llvm2aa}  is a tool which reads in LLVM byte-code (see http://www.llvm.org for
details about LLVM) and produces \Aa code which can then be further used
to produce VHDL using the AhirV2 tool chain developed at IIT Bombay.

\section{Synopsys}

The typical usage of the tool is 
\begin{verbatim}
llvm2aa [-modules=<listfile>] [-storageinit] bytecode.o > bytecode.aa
\end{verbatim}
The generated \Aa code is sent to {\bf stdout} and all informational
messages are sent to {\bf stderr}.  On success, the tool returns 0.

The options:
\begin{itemize}
\item {\bf -modules=listfile} : Specify the list of functions in the bytecode
which should be converted to \Aa.   The names of these functions should be
listed in the text-file listfile. If absent, all functions
are converted.
\item {\bf -storageinit} :  Storage objects in the llvm bytecode
are explicitly initialized in the generated \Aa code.   An initializer
routine {\bf global\_storage\_initializer\_} is instantiated in
the \Aa code for this purpose.
\end{itemize}

\subsection{{\bf AaLinkExtMem}}

This tool takes a list of \Aa files, elaborates the program,
does memory space decomposition.  The externally visible memory space is
linked in one of two ways: either it is assumed to be external
and all accesses to it are routed out of the \Aa program,
or it is assumed to be internal and assumed to consist of
a memory object (an array of bytes).  External pointer dereferences
are handled as if they are directed at this memory object.
\begin{verbatim}
AaLinkExtMem [-I n] [-E obj-name] bytecode.o > bytecode.aa
\end{verbatim}
The generated \Aa code is sent to {\bf stdout} and all informational
messages are sent to {\bf stderr}.  On success, the tool returns 0.

The options:
\begin{itemize}
\item {\bf -I n}: specifies that external references to memory
are to be mapped as if they are to an internal object whose size
is $n$ bytes.
\item {\bf -E obj-name} : specifies that the object to which
external references are mapped is to be named obj-name.
\end{itemize}
It is better if you use the {\bf -I} and {\bf -E} options.  

If the {\bf -I} option is not used, then all external memory
references are routed out of the \Aa program through pipes.
In this case, if the \Aa compiler determines that there is some pointer 
in the program which can point
to both internal and external memory, then this will be
declared as an error!




\subsection{{\bf Aa2VC}}

This tool takes a list of \Aa programs and converts them
to a \vC description. 
\begin{verbatim}
Aa2VC [-O] [-C] [-I obj-name] file1.aa file2.aa ...  > result.vc
\end{verbatim}
The generated \vC code is sent to {\bf stdout} and all informational
messages are sent to {\bf stderr}.  On success the tool returns 0.

The options:
\begin{itemize}
\item {\bf -O} : if used, sequential statement blocks are parallelized
by doing dependency analysis.
\item {\bf -C} : if used, a C stub is created for every module that
is not called from within the system.  These stubs can be used to
interface to a VHDL simuator (or even drive hardware) to verify
the VHDL code generated by downstream tools.
\item {\bf -I obj-name} : if specified, all external memory references
are considered as being directed at the storage object named obj-name.
\end{itemize}

\subsection{\bf vc2vhdl}

Takes a collection of \vC descriptions and converts them to
VHDL.
\begin{verbatim}
vc2vhdl [-O] [-C] -t foo [-t bar -t bar2 ...]\
         -f file1.vc -f file2.vc ...  > system.vhdl
\end{verbatim}

The options:
\begin{itemize}
\item {\bf -t} : to specify the modules which are to be 
accessible from the ports of the generated VHDL system.
Multiple top-level modules can be specified in this way.
\item {\bf -f} : specifies the \vC files to be analyzed. 
Multiple \vC files may be specified.
\item {\bf -O} : optimize the generated VHDL by compacting
the control-path.  This does not change the resulting
hardware, but makes the generated VHDL file smaller.
\item {\bf -C} : the VHDL code has a system test bench which
interfaces to foreign code using a VHPI/Modelsim-FLI interface.
If this is not specified, the  generated test bench simply
instantiates the system and starts all top-level modules
off (you will need to fill in your own test bench here).
The C testbench is usually easier to write (it probably
already exists in the form of the original program).
\end{itemize}

The tool performs concurrency analysis to determine operations which
can be mapped to the same physical operator without the need for
arbitration.  It also instantiates separate memory subsystems for
the disjoint memory spaces (in practice many of the memory spaces
are small and are converted to register banks).

\subsection{Miscellaneous: {\bf vcFormat} and {\bf vhdlFormat}}

The outputs produces by {\bf Aa2VC} and {\bf vc2vhdl} are
not well formatted.  One can format \Aa and \vC files
using  {\bf vcFormat} as follows
\begin{verbatim}
vcFormat < unformatted-vc/aa-file  > formatted-vc/aa-file
\end{verbatim}
and similarly use {\bf vhdlFormat} to format generated
VHDL files.


\section{An example} \label{sec:Example}

Let us revisit the simple example considered in
the first section:
\begin{verbatim}
int add(int a, int b)
{
        int c = (a+b);
        return(c);
}
\end{verbatim}
We wish to generate a circuit which {\em implements}
the specification implied by this program.

We convert the program to LLVM byte code using
the {\bf clang} compiler (www.llvm.org)
\begin{verbatim}
      clang -std=gnu89 -emit-llvm -c add.c
\end{verbatim}
This produces a binary file {\bf add.o} which is
the LLVM byte-code.  To make the byte-code human
readable, we dis-assemble it using an LLVM utility
\begin{verbatim}
     llvm-dis add.o
\end{verbatim}
This is what the LLVM assembly code looks like
\begin{verbatim}
; ModuleID = 'add.o'
target datalayout = "e-p ...... "
target triple = "i386-pc-linux-gnu"

define i32 @add(i32 %a, i32 %b) nounwind {
  %1 = alloca i32, align 4
  %2 = alloca i32, align 4
  %c = alloca i32, align 4
  store i32 %a, i32* %1, align 4
  store i32 %b, i32* %2, align 4
  %3 = load i32* %1, align 4
  %4 = load i32* %2, align 4
  %5 = add nsw i32 %3, %4
  store i32 %5, i32* %c, align 4
  %6 = load i32* %c, align 4
  ret i32 %6
}
\end{verbatim}
To get to this point, we could have used several
optimizations which are available in the LLVM frame-work.
But we work with the unoptimized version to illustrate
the storage decomposition which is carried out by
the AhirV2 tools.

The LLVM byte-code is our starting point.  We first convert it
to \Aa.
\begin{verbatim}
llvm2aa add.o | vcFormat > add.o.aa
\end{verbatim}
This produces an \Aa program
\begin{verbatim}
// Aa code produced by llvm2aa (version 1.0)
$module [add]
// arguments
$in (a : $uint<32> b : $uint<32> )
$out (ret_val__ : $uint<32>)
$is
{
  $storage stored_ret_val__ : $uint<32>
  $branchblock [add]
  {
    //begin: basic-block bb_0
    $storage iNsTr_0 : $uint<32>
    $storage iNsTr_1 : $uint<32>
    $storage c : $uint<32>
    iNsTr_0 := a
    iNsTr_1 := b
    // load
    iNsTr_4 := iNsTr_0
    // load
    iNsTr_5 := iNsTr_1
    iNsTr_6 := (iNsTr_4 + iNsTr_5)
    c := iNsTr_6
    // load
    iNsTr_8 := c
    stored_ret_val__ := iNsTr_8
    $place [return__]
    $merge return__ $endmerge
    ret_val__ := stored_ret_val__
  }
}
\end{verbatim}

Now, this \Aa code is converted to a virtual circuit \vC representation.
\begin{verbatim}
     Aa2VC -O add.o.aa | vcFormat > add.o.aa.vc
\end{verbatim}
The virtual circuit representation is a bit too verbose to reproduce entirely
here, but we show some critical fragments
\begin{verbatim}
$module [add] 
{
  $in a:$int<32> b:$int<32>
  $out ret_val__:$int<32>
  $memoryspace [memory_space_0] 
  {
    $capacity 1
    $datawidth 32
    $addrwidth 1
    // ret-val is kept here
    $object [xxaddxxstored_ret_val__] : $int<32>
  }
  $memoryspace [memory_space_1] 
  {
    $capacity 1
    $datawidth 32
    $addrwidth 1
    // a is kept here.
    $object [xxaddxxaddxxiNsTr_0] : $int<32>
  }
  $memoryspace [memory_space_2] 
  {
    $capacity 1
    $datawidth 32
    $addrwidth 1
    // b is kept her
    $object [xxaddxxaddxxiNsTr_1] : $int<32>
  }
  $memoryspace [memory_space_3] 
  {
    $capacity 1
    $datawidth 32
    $addrwidth 1
    // c is kept here.
    $object [xxaddxxaddxxc] : $int<32>
  }
  $CP 
  {
     // a control-flow petri-net..  verbose..
  }
  // end control-path
  $DP 
  {
     // wires and operators.
  }

   // links between CP and DP
}
\end{verbatim}
The important points to note are that the stored objects a,b,c and
ret\_val\_\_ are mapped to different memory spaces.  Thus, the
chief difference between a \vC description and a processor is 
that the \vC program partitions storage into small units which
are accessed only by operators that need them. 

Finally, we take the \vC description and convert it to
VHDL
\begin{verbatim}
vc2vhdl -t add -f add.o.aa.vc | vhdlFormat > system.vhdl
\end{verbatim}
This produces a VHDL implementation of the system with
{\bf add} marked as a top-level module.  The VHDL that
is produced is too voluminous to reproduce here, but
the top-level system entity is
\begin{verbatim}
entity test_system is  -- system
  port (--
    add_a : in  std_logic_vector(31 downto 0);
    add_b : in  std_logic_vector(31 downto 0);
    add_ret_val_x_x : out  std_logic_vector(31 downto 0);
    add_tag_in: in std_logic_vector(0 downto 0);
    add_tag_out: out std_logic_vector(0 downto 0);
    add_start : in std_logic;
    add_fin   : out std_logic;
    clk : in std_logic;
    reset : in std_logic); --
  --
end entity;
\end{verbatim}
There are ports corresponding to the arguments of the top-level module,
and the add\_start/add\_fin is a handshake pair.  One sets of the inputs,
starts the system and waits until the fin is asserted.  After the fin
is asserted, one has the return value of at the appropriate port.

\section{External and internal memory spaces}

Consider the following C program:
\begin{verbatim}
int main(int* b)
{
   int q[2];
   q[0] = *b;
   q[1] = q[0];
   return(q[1]);
}
\end{verbatim}
When this program is mapped to a circuit, we identify 
two distinct memory spaces, one which contains the
array $q$ and the other corresponding to the 
external world (the one referred to by the pointer $b$).
Where is the external memory physically located?  In 
the AhirV2 flow, we can either locate it outside
the system or inside the system which is being 
described by this program.

If the external memory is to be placed outside, then
accesses to it from within the system must be routed
outside the system.  On the other hand if it is
to be placed inside, a storage object corresponding 
to it must be created and all accesses to the external
memory must be directed at this storage object.
Further, the external world must have a mechanism for
accessing this storage object.

Both options are supported in the AhirV2 flow through
the utility {\bf AaLinkExtMem} described earlier.


\subsection{Keeping the external memory outside the system}

For this example, if you want to keep the external memory
outside, you will have to go through the following sequence
\begin{verbatim}
# first use clang (or llvm-gcc) to generate llvm-byte-code
clang -std=gnu89 -emit-llvm -c foo.c
#
# disassemble so that you can make sense of the llvm bc.
llvm-dis foo.o
#
# OK, now take the llvm byte code
# and generate an Aa description.
# use the storageinit option to initialize
# global storage.
# (the pipe to vcFormat is to prettify the output)
 llvm2aa -storageinit foo.o | vcFormat > foo.o.aa
#
#
# Do an Aa -> Aa transformation: map external
# memory outside..
AaLinkExtMem foo.o.aa | vcFormat > foo.o.memlinked.ExternalOutside.aa
#
# Now take the Aa code and generate a virtual
# circuit..
# the -O flag does dependency analysis in straight-line
# code and parallelizes it.
#
Aa2VC -O foo.o.memlinked.ExternalOutside.aa | vcFormat\
                 > foo.o.memlinked.ExternalOutside.aa.vc
#
# finally, generate vhdl from the vc description.  Note that
# you will have to mark the module foo as well as the
# extmem_store_32/load_32 modules as top-level modules
# so that it is possible for the outside world to serve
# requests made from inside.
#
vc2vhdl -O -t foo -t extmem_store_32 -t extmem_load_32\ 
         -f foo.o.memlinked.ExternalOutside.aa.vc | vhdlFormat\
              > foo_o_aa_memlinked_external_outside_vc.vhdl

\end{verbatim}

If you look at the generate top-level VHDL entity, its ports
will be
\begin{verbatim}
entity test_system is  -- system
  port (--
    foo_b : in  std_logic_vector(31 downto 0);
    foo_ret_val_x_x : out  std_logic_vector(31 downto 0);
    foo_tag_in: in std_logic_vector(0 downto 0);
    foo_tag_out: out std_logic_vector(0 downto 0);
    foo_start : in std_logic;
    foo_fin   : out std_logic;
    clk : in std_logic;
    reset : in std_logic;
    extmem_read_address_32_pipe_read_data: out std_logic_vector(31 downto 0);
    extmem_read_address_32_pipe_read_req : in std_logic_vector(0 downto 0);
    extmem_read_address_32_pipe_read_ack : out std_logic_vector(0 downto 0);
    extmem_read_data_32_pipe_write_data: in std_logic_vector(31 downto 0);
    extmem_read_data_32_pipe_write_req : in std_logic_vector(0 downto 0);
    extmem_read_data_32_pipe_write_ack : out std_logic_vector(0 downto 0);
    extmem_write_address_32_pipe_read_data: out std_logic_vector(31 downto 0);
    extmem_write_address_32_pipe_read_req : in std_logic_vector(0 downto 0);
    extmem_write_address_32_pipe_read_ack : out std_logic_vector(0 downto 0);
    extmem_write_data_32_pipe_read_data: out std_logic_vector(31 downto 0);
    extmem_write_data_32_pipe_read_req : in std_logic_vector(0 downto 0);
    extmem_write_data_32_pipe_read_ack : out std_logic_vector(0 downto 0)); --
  --
end entity;
\end{verbatim}
The external memory read and write address and data are clearly
visible.   The outside world is responsible for serving the
read/write requests made from the inside.

\subsection{Keeping the external memory inside the system}

For this example, if you want to keep the external memory
outside, you will have to go through the following sequence
\begin{verbatim}
# use clang (or llvm-gcc) to generate llvm-byte-code
clang -std=gnu89 -emit-llvm -c foo.c
#
# disassemble so that you can make sense of the llvm bc.
llvm-dis foo.o
#
# OK, now take the llvm byte code
# and generate an Aa description.
# use the storageinit option to initialize
# global storage.
# (the pipe to vcFormat is to prettify the output)
 llvm2aa -storageinit foo.o | vcFormat > foo.o.aa
#
#
# Do an Aa -> Aa transformation: map external
# memory to a storage area inside the system...
# -I 1024 says that the amount of memory that will be
# referred to is 1024 bytes.
# -E mempool says that the storage object corresponding
# to external memory is named mempool.
AaLinkExtMem  -I 1024 -E mempool foo.o.aa | vcFormat\
               > foo.o.memlinked.ExternalInside.aa
#
# Now take the Aa code and generate a virtual
# circuit..
# the -O flag does dependency analysis in straight-line
# code and parallelizes it.
# the -I mempool option says that external memory is
# to be mapped inside the system to object mempool..
#
Aa2VC -O -I mempool foo.o.memlinked.ExternalInside.aa\
           | vcFormat > foo.o.memlinked.ExternalInside.aa.vc
#
# finally, generate vhdl from the vc description.
# note that you will have to mark mem_load__ and mem_store__
# as top-level modules, so that the external world can
# access its memory pool inside the system.
#
vc2vhdl -O -t foo -t mem_load__ -t mem_store__ \ 
     -f foo.o.memlinked.ExternalInside.aa.vc\
       | vhdlFormat > foo_o_aa_memlinked_external_inside_vc.vhdl
\end{verbatim}

The generated top-level VHDL entity has the following
ports:
\begin{verbatim}
entity test_system is  -- system
  port (--
    foo_b : in  std_logic_vector(31 downto 0);
    foo_ret_val_x_x : out  std_logic_vector(31 downto 0);
    foo_tag_in: in std_logic_vector(0 downto 0);
    foo_tag_out: out std_logic_vector(0 downto 0);
    foo_start : in std_logic;
    foo_fin   : out std_logic;
    mem_load_x_x_address : in  std_logic_vector(31 downto 0);
    mem_load_x_x_data : out  std_logic_vector(7 downto 0);
    mem_load_x_x_tag_in: in std_logic_vector(0 downto 0);
    mem_load_x_x_tag_out: out std_logic_vector(0 downto 0);
    mem_load_x_x_start : in std_logic;
    mem_load_x_x_fin   : out std_logic;
    mem_store_x_x_address : in  std_logic_vector(31 downto 0);
    mem_store_x_x_data : in  std_logic_vector(7 downto 0);
    mem_store_x_x_tag_in: in std_logic_vector(0 downto 0);
    mem_store_x_x_tag_out: out std_logic_vector(0 downto 0);
    mem_store_x_x_start : in std_logic;
    mem_store_x_x_fin   : out std_logic;
    clk : in std_logic;
    reset : in std_logic); --
  --
end entity;
\end{verbatim}

The system provides a memory load and memory store
port to the external world (through the mem\_load.. and mem\_store..)
ports.

\section{Handling complex programs}

This example is trivial.  Real, non-trivial programs can be
mapped in this manner.  Currently, there are only
two restrictions
\begin{itemize}
\item No recursion, no cycles in the call-graph of the original
program.
\item No function pointers.
\end{itemize}

The AhirV2 flow produces a modular system, and by default,
produces one VHDL entity for every function in the original
program.  Further, the concept of message pipes is native
to the \Aa and \vC descriptions.  Thus, it is easy to
map parallel programs or programs consisting of multiple
concurrent processes to hardware.

\section{To be continued}

This document is a work-in-progress.  More details and
related documentation will be added shortly.

The examples folder contains some explanations which might
make things easier to understand.

\begin{thebibliography}{99}
\bibitem{ref:dsd2010}
Sameer D. Sahasrabuddhe, Sreenivas Subramanian, Kunal P. Ghosh, Kavi Arya, Madhav P. Desai, 
"A C-to-RTL Flow as an Energy Efficient Alternative to Embedded Processors in Digital Systems," 
DSD, pp.147-154, 2010 13th Euromicro Conference on Digital 
System Design: Architectures, Methods and Tools, 2010
\end{thebibliography}
\end{document}
