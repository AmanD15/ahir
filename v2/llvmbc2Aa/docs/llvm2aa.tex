\documentclass{article}

\title{About {\bf llvm2aa}}
\author{Madhav Desai \\ Department of Electrical Engineering \\ Indian Institute of Technology \\
	Mumbai 400076 India}

\newcommand{\Aa}{{\bf Aa}~}
\newcommand{\vC}{{\bf vC}~}

\begin{document}
\maketitle

\section{Introduction}

{\bf llvm2aa}  is a tool which reads in LLVM byte-code (see http://www.llvm.org for
details about LLVM) and produces \Aa code which can then be further used
to produce VHDL using the AhirV2 tool chain developed at IIT Bombay.

\section{Synopsys}

The typical usage of the tool is 
\begin{verbatim}
llvm2aa [-modules=<listfile>] [-storageinit] [llvm-passes] bytecode.o > bytecode.aa
\end{verbatim}
The generated \Aa code is sent to {\bf stdout} and all informational
messages are sent to {\bf stderr}.  On success, the tool returns 0.

The options:
\begin{itemize}
\item {\bf --modules=listfile} : Specify the list of functions in the bytecode
which should be converted to \Aa.   The names of these functions should be
listed in the text-file listfile.  One function name must be present
on each line (the {\bf exact} function name should be provided on the line, without leading
or trailing spaces).  If this option is not specified, all functions
are converted.
\item {\bf --storageinit} :  Storage objects in the llvm bytecode
are explicitly initialized in the generated \Aa code.   An initializer
routine {\bf global\_storage\_initializer\_} is instantiated in
the \Aa code for this purpose.
\item {\bf -pipedepths=depthfile} : Specify the maximum depths of
pipes in the generated Aa program. The file ``depthfile'' is a list
of pairs (one pair per line).  The first element of the pair specifies
a pipe name, and the second, its maximum depth.   The default depth
of any pipe is $1$.
\item {\bf llvm2aa} uses the LLVM compiler
infrastructure to perform LLVM byte-code optimizations.  A large list
of these optimizations is available through the llvm2aa command-line.
For more details, see LLVM documentation at http://www.llvm.org.
\end{itemize}

\section{Limitations}

Several LLVM byte-code constructs are not supported.  Most 
importantly:
\begin{itemize}
\item Function pointers are not supported.
\item Functions with a variable number of arguments are not 
supported.
\item Calls to LLVM intrinsics are just passed through to the
output \Aa file.  The \Aa file will then contain calls
to these intrinsics without there being a corresponding
module declaration in the \Aa file.
\item If the LLVM byte-code has cycles in its call graph,
then the code is translated, but will create an error
in downstream \Aa analysis and transformation tools.
\item System calls made from the \Aa code are simply
passed through and would need to be supplied as an
\Aa library in order to perform downstream analysis and
transformation.
\item The LLVM integer, floating-point, array, structure,
vector and void types are the only ones currently supported.
\item The LLVM indirect-branch, invoke, unwind and unreachable
instructions will not be supported.
\item The LLVM division and remainder instructions are currently
not supported. 
\item LLVM vector instructions are currently not supported.
\item LLVM aggregate instructions (extractvalue, insertvalue)
are currently not supported, but will be supported in the
near future.
\end{itemize}

\section{Examples}

Let us start with the following C program, kept
in file ``add.c''.
\begin{verbatim}
int add(int a, int b)
{
   int c = (a+b);
   return(c);
}
\end{verbatim}

We will first need to compile this program down to 
LLVM byte code.  For this, we use the {\bf clang}
compiler (http://www.clang.org)
\begin{verbatim}
clang -std=gnu89 -emit-llvm -c add.c
\end{verbatim}
This produces an LLVM byte-code file {\bf add.o},
which contains a compiled version of the function
in the file shown above.
This is our starting point.

We use 
\begin{verbatim}
llvm2aa  -storageinit add.o > add.o.aa 
\end{verbatim}
to generate an \Aa version of the LLVM bytecode.
All functions in the LLVM bytecode will be translated
and initial values of globally declared objects will
be ignored.
The \Aa file that is produced is
\begin{verbatim}
// Aa code produced by llvm2aa (version 1.0)
$module [add] 
$in (a : $uint<32> b : $uint<32> )
$out (ret_val__ : $uint<32>)
$is 
{
  $storage stored_ret_val__ : $uint<32>
  $branchblock [body] 
  {
    //begin: basic-block bb_0
    $storage iNsTr_0_alloc : $uint<32>
    $storage iNsTr_1_alloc : $uint<32>
    $storage c_alloc : $uint<32>
    iNsTr_0 := @(iNsTr_0_alloc)
    iNsTr_1 := @(iNsTr_1_alloc)
    c := @(c_alloc)
    ->(iNsTr_0) := a
    ->(iNsTr_1) := b
    // load 
    iNsTr_4 := ->(iNsTr_0) 
    // load 
    iNsTr_5 := ->(iNsTr_1) 
    iNsTr_6 := (iNsTr_4 + iNsTr_5)
    ->(c) := iNsTr_6
    // load 
    iNsTr_8 := ->(c) 
    stored_ret_val__ := iNsTr_8
    $place [return__]
    $merge return__ $endmerge
    ret_val__ := stored_ret_val__ 
  }
}
\end{verbatim}

\end{document}
