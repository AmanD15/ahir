\documentclass[a4paper,12pt]{article}
\usepackage{geometry}
\geometry{margin=1in}

\usepackage{subfigure}
\usepackage{graphicx}
\usepackage{amsmath}
\usepackage{times}
\usepackage{tikz}

\newtheorem{statement}{Statement}[section]
\newtheorem{condition}{Condition}

\newcommand{\SYM}[1]{$\operatorname{#1}$}

\title{Eliminating registers at the outputs of data-path operators}
\date{}

\begin{document}

\maketitle

\section{Introduction}

Every operator in the data-path has a register at its output, that
takes at least one clock cycle before signalling completion. In some
cases it may be possible to eliminate this register. The result is a
combinational path from the input of the operator to the output
register of the next operator. For example,
Figure~\ref{fig:operator-sequence} shows a sequence of operations
($A\longrightarrow B\longrightarrow C$) where ovals represent
combinational logic and boxes represent registers. If the register
\SYM{reg\_B} is removed, the logic for \SYM{op\_B} and \SYM{op\_C} is
combined into one large combinational path as shown in
Figure~\ref{fig:combinational-short-circuit}.

\begin{figure}[htb]
  \centering
  \subfigure[Operator logic separated by registers.]{
    \label{fig:operator-sequence}
    \includegraphics[scale=0.75]{images/operator-sequence.pdf}}\\
  \vspace{0.25in}
  \subfigure[Combinational path lengthened by eliminating register
  reg\_B.]{
    \label{fig:combinational-short-circuit}
    \includegraphics[scale=0.75]{images/combinational-short-circuit.pdf}}
  \caption{Eliminating registers at operator outputs.}
  \label{fig:register-elimination}
\end{figure}
  
For this transformation to be correct, we must guarantee that the
absence of the register does not result in invalid data at the input
of other nodes in the data-path. When  \SYM{reg\_A} is updated,
the new value at the output of \SYM{op\_B} is immediately available at
the input of \SYM{op\_C}, in the absence of  \SYM{reg\_B}. This
may result in an incorrect computation at \SYM{C}, if the actual
operation was meant to use the old value stored in \SYM{reg\_B}. In
the following sections, we propose a set of necessary and/or
sufficient conditions that guarantee that the output of all
computations remains valid after eliminating a register.

\section{Terminology}
\label{sec:terminology}

\begin{description}
\item [Register definition:] An ``operation''(?) in the control-path
  that results in updating the value of a register in the data-path is
  called a \emph{definition} of that register. The set of such
  definitions of a register $A$ is written as $D_A =
  \{d^0_A,d^1_A,\ldots\}$.
\item [Register use:] A definition $d^i_B$ is said to \emph{use}
  register $A$ when the value assigned to $B$ is computed in a
  ``non-trivial''(?) manner from the value of $A$. This is written as
  $A\mapsto d^i_B$.
\item[Predecessor:] A register $A$ is termed as the \emph{predecessor}
  of a register $B$ if the value of $A$ is used in some definition of
  $B$, denoted by the operator ``$\longrightarrow$''. Also, $B$ is
  termed as a \textbf{successor} of $A$.
  
  \[A \longrightarrow B \iff \exists d^i_B \mbox{ such that } A\mapsto
  d^i_B\]
\end{description}

\section{Values latched from external sources}
\label{sec:external-data}

Some operators in the data-path acquire values from the environment,
such as memory access operators and input/output ports. The
environment is free to change the value after that particular
invocation of the operator is done --- the register at the output of
the operator is required to maintain it until it has been used by
other operations in the data-path. Hence that register cannot be
eliminated.

\begin{condition}
  \label{condition:external-data}
  Every definition of the candidate register must assign a value
  computed from an expression that consists of only constants and
  values stored in other registers in the data-path.
\end{condition}

\section{Registers with conditional definitions}
\label{sec:conditional-definitions}

In general, a data-path operator may have multiple control inputs and
multiple data outputs (registers). When the operator is invoked by a
particular control input, the state machine of the operator determines
whether a particular register is defined. Each such definition may
assign a different value to the same register. This behaviour results
in a multiplexer that selects the result of multiple combinational
circuits. If such a register is eliminated, the internal multiplexing
(which is dynamic) will drive invalid data at the output.

\begin{condition}
  \label{condition:conditional-definitions}
  Every definition of the candidate register must assign the same
  value to the register.
\end{condition}

\begin{figure}[htb]
  \centering
  \includegraphics[scale=0.3]{images/phi-function.pdf}
  \caption{Multiple definitions in a $\phi$-function.}
  \label{fig:phi-function}
\end{figure}

For example, the $\phi$-function in the AHIR data-path has two control
inputs, two data inputs and a single output register, as shown in
Figure~\ref{fig:phi-function}. When definition $d^0_M$ (or
respectively $d^1_M$) is invoked, the corresponding data-input $D_0$
(respectively $D_1$) is assigned to the output register. Eliminating
the output register will make it impossible to forward only one of the
two data inputs to the output. Hence, the register at the output of a
$\phi$-function cannot be eliminated.

\section{Definition-Use sequences}
\label{sec:def-use-chains}

A register $R$ that satisfies
Condition~\ref{condition:conditional-definitions} is assigned a value
generated by a single combinational circuit. The output of the circuit
changes when any of its input registers is updated. In the absence of
the register $R$, this value is immediately visible to the uses of
$R$. Hence $R$ can be eliminated only if the control-path guarantees
that when any of the inputs change, $R$ is always updated before it is
used.

\begin{condition}
  \label{condition:def-use-chains}
  For every predecessor $P$ and successor $S$ of the candidate
  register $B$, if there is a path from any definition of $P$ to any
  definition of $S$ that uses the value of $B$, then that path must
  pass through some definition of $B$.
\end{condition}

\begin{figure}[htb]
  \centering
  \includegraphics[scale=0.3]{images/branch-conditional-definition.pdf}
  \caption{Conditional definitions in the presence of branches.}
  \label{fig:branch-conditional-definition}
\end{figure}

For example, consider the output of operator $B$ shown in
Figure~\ref{fig:branch-conditional-definition}. Definition $d^0_B$ is
executed only if the appropriate branch is taken in the control-path
after definition $d^0_P$. The value of $B$ is used by definition
$d^1_M$, which occurs along the same path in the control-path. Thus,
every path from a definition of $P$ to a definition of $M$ that uses
$B$ (there is only one such path) passes through a definition of $B$.
Hence $B$ satisfies this condition for elimination.

\begin{figure}[htb]
  \centering
  \includegraphics[scale=0.3]{images/elimination-in-a-pipeline.pdf}
  \caption{Registers in a pipeline.}
  \label{fig:pipeline-registers}
\end{figure}

Figure~\ref{fig:pipeline-registers} shows a pipeline that invokes
$d_A$, $d_B$ and $d_C$ in parallel, where $A\mapsto d_B$ and $B\mapsto
d_C$. Register $B$ does not satisfy
Condition~\ref{condition:def-use-chains} since there is a path from
$d_A$ to $d_C$ which does not pass through $d_B$. Hence register $B$
cannot be eliminated.

\section{Cycles in the data-path}
\label{sec:data-path-cycles}

Every cycle in the data-path must have at least one register in order
to prevent a combinational loop.

\begin{condition}
  \label{condition:data-path-cycles}
  If the candidate register occurs in a cycle in the data-path, then
  it must not be the only register in that cycle.
\end{condition}
\end{document}
