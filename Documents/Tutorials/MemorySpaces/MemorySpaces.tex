\documentclass{beamer}
\usepackage{epsfig}
\title{Memory Spaces in AHIRV2}
\author{Madhav P. Desai}
\begin{document}
\maketitle



\frame[containsverbatim]{\frametitle{Storage Variables and Memory Spaces}

\begin{itemize}
\item Storage variables are grouped into memory spaces.
\item Each memory space is implemented in hardware by a memory sub-system. 
\item Memory spaces are identified by doing a static aliasing analysis.
\end{itemize}
}


\frame[containsverbatim]{\frametitle{Aliasing analysis}
\begin{verbatim}
int A, B, C, D; // storage objects.
int foo()
{
   int ret_val = 0;
   int* p1 = &A;
   ret_val += *p1;
   int *p2 = &B;
   ret_val += *p2;
   int *p3 = &C;
   ret_val += *p3;
   p3 = &D;
   ret_val += *p3;
   return(ret_val);
}
\end{verbatim}
}


\frame[containsverbatim]{\frametitle{Groupings}

\begin{itemize}
\item The memory spaces are $\{A\}$, $\{ B\}$, $\{ C, D \}$.
\item $C$ and $D$ are grouped together because pointer $p3$ can
point to either $C$ or $D$.
\end{itemize}

}

\frame[containsverbatim]{\frametitle{Tool-flow for identifying memory-spaces.}

\begin{verbatim}
AaLinkExtMem -I 1 -E mempool prog1.aa prog2.aa | vcFormat > linked.aa
Aa2VC -E mempool  linked.aa ....
\end{verbatim}

}


\frame[containsverbatim]{\frametitle{AaLinkExtMem}
\begin{verbatim}
AaLinkExtMem -I 1 -E mempool prog1.aa prog2.aa | vcFormat > linked.aa
\end{verbatim}
\begin{itemize}
\item -I 1 : introduce a dummy storage variable of size 1 byte as target for unclassified
pointer dereferences.
\item -E mempool:  this dummy variable is called mempool.
\item Use the -E mempool option with Aa2VC to pass this information to vC and further to VHDL.
\end{itemize}
}

\frame[containsverbatim]{ \frametitle{Example}
}
\end{document}
