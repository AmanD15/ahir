\documentclass{beamer}
\usepackage{epsfig}
\title{Describing Pipelines}
\author{Madhav P. Desai}
\begin{document}
\maketitle



\frame[containsverbatim]{\frametitle{Two ways of describing pipelines}

\begin{itemize}
\item with distinct threads.
\item by pipelining a loop.
\end{itemize}
}


\frame[containsverbatim]{\frametitle{An example}
\begin{itemize}
\item We want to compute $d(k) = (a(k)+b(k)) \times c(k)$ for a sequence of 
numbers. 
\item The system should accept a sequence of numbers $\{a(k),b(k),c(k)\}$
and produce the sequence $\{ d(k) \}$.
\end{itemize}
}


\frame[containsverbatim]{\frametitle{As separate threads}
\begin{verbatim}
void Stage_0()
{
  while(1)
  {
    float a = read_float32("a_pipe");
    float b = read_float32("b_pipe");
    float c = read_float32("c_pipe");
    write_float32("c_forward_pipe", c);
    write_float32("s0_result", a+b);
  }
}
void Stage_1()
{
  while(1)
  {
    float c = read_float32("c_forward_pipe");
    float aSb = read_float32("s0_result");
    write_float32("d_pipe", aSb * c);
  }
}
\end{verbatim}
}

\frame[containsverbatim]{\frametitle{In a single loop}
\begin{verbatim}
void Daemon()
{
  while(1)
  {
    float a = read_float32("a_pipe");
    float b = read_float32("b_pipe");
    float c = read_float32("c_pipe");
    write_float32("d_pipe", (a+b)*c);
  }
}
\end{verbatim}
}


\frame[containsverbatim]{\frametitle{Loop Pipelining}
\begin{itemize}
\item Loop pipelining involves executing multiple iterations of a
loop at the same time.
\item While doing this, all dependencies must be taken care of.
\begin{itemize}
\item Operation order.
\item Data dependency.
\item Memory accesses.
\item Pipe accesses.
\end{itemize}
\end{itemize}
}

\frame[containsverbatim]{ \frametitle{Example}
}
\end{document}
