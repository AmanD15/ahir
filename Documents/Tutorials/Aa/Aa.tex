\documentclass{beamer}
\usepackage{epsfig}
\title{An introduction to the {\bf Aa} language}
\author{Madhav P. Desai}
\begin{document}
\maketitle



\frame[containsverbatim]{\frametitle{The {\bf Aa} language}

\begin{itemize}
\item Serves as an intermediate representation in the AHIR-V2 flow.
\item Control-flow (imperative) language with support for parallelism
and branching.
\end{itemize}
}


\frame[containsverbatim]{\frametitle{A simple program in {\bf Aa}}
\begin{verbatim}
$module [maxOfTwo] $in (a: $uint<32> b:$uint<32>)
   $out (c: $uint<32>) $is
{
   c := ( $mux (a > b) a b )
}
\end{verbatim}
}


\frame[containsverbatim]{\frametitle{Program-structure in {\bf Aa} }
\begin{verbatim}
aA_Program :=  ( aA_Module | 
                 aA_Object_Declaration | 
                 aA_Named_Type_Declaration)*
\end{verbatim}
}

\frame[containsverbatim]{\frametitle{A module}
\begin{verbatim}
aA_Module := $module [identifier]
                $in ( aA_Module_Args) 
                $out (aA_Module_Args) 
{
        aA_Object_Declaration*
        aA_Atomic_Statement_Sequence
}
\end{verbatim}
}


\frame[containsverbatim]{\frametitle{{\bf Aa} Types}

{\bf Aa} provides a comprehensive set of types.
\begin{itemize}
\item {\em Unsigned integers} 
\begin{verbatim}
$uint<1>, $uint<32> etc.
\end{verbatim}
\item {\em Signed integers}   
\begin{verbatim}
$int<1>, $int<32> etc.
\end{verbatim}
\item {\em Sized floats}      
\begin{verbatim} 
$float<8,32>, $float<11,52>
\end{verbatim}
\item {\em Pointers} : 
\begin{verbatim}
$pointer<$uint<32>> etc..
\end{verbatim}
\item {\em Arrays}:  
\begin{verbatim}
$array[32][4] $of $uint<32> etc.
\end{verbatim}
\item {\em Records}: 
\begin{verbatim}
$record $uint<32> $uint<1>
\end{verbatim}
\item {\em Named Records}: 
\begin{verbatim}
$record [MyRec] $pointer<MyRec>) $uint<32>
\end{verbatim}
\end{itemize}
}

\frame[containsverbatim]{\frametitle{{\bf Aa} Objects}
\begin{itemize}
\item Storage objects:
\begin{verbatim}
$storage A: $array [1024] $of $uint<32>
\end{verbatim}
\item Constant objects:
\begin{verbatim}
$constant A: $uint<4> := _b0011
$constant B: $uint<32> := _hf0f0f0f0
$constant C: $float<8,23> := _f2.3465e+0
\end{verbatim}
\item Pipe objects:
\begin{verbatim}
$pipe A: $uint<32> $depth 4
$lifo $pipe B: $uint<8> $depth 8
\end{verbatim}
\item Implicit Variables:  these are defined by
statements:
\begin{verbatim}
a := (A + B)
\end{verbatim}
They are also called {\em static-single-assignment}
or SSA variables.
\end{itemize}
}


\frame[containsverbatim]{\frametitle{{\bf Aa} Expressions}
\begin{itemize}
\item Constants: 
\begin{verbatim}
_b00011
_habf1
\end{verbatim}
\item Simple references:
\begin{verbatim}
a
\end{verbatim}
\item Array references:
\begin{verbatim}
a[0][1] 
a[(I+1)][J][K] 
\end{verbatim}
\item Unary expressions:
\begin{verbatim}
(<op> expr)
e.g.  (~ a)
\end{verbatim}
\item Binary expressions:
\begin{verbatim}
(a <op> B)
<op> can be +,-,*,/,<<,>>,|,&,~|,~&,^,~^
            ==,!=,>,>=,<,<=
\end{verbatim}
\end{itemize}
}

\frame[containsverbatim]{\frametitle{{\bf Aa} More Expressions}
\begin{itemize}
\item Ternary expressions:
\begin{verbatim}
($mux <test-expr> <if-expr> <else-expr> )
e.g.  ($mux (a > 0) (b+1) (b-1))
\end{verbatim}
\item Concatenation expression:
\begin{verbatim}
(a && b)
\end{verbatim}
\item Bit-select expression:
\begin{verbatim}
(a [] I)
\end{verbatim}
\item Address-of expression:
\begin{verbatim}
@(a)
@(a[I])
\end{verbatim}
\item Pointer-dereference expression:
\begin{verbatim}
->(ptr)
\end{verbatim}
If it appears on the left-hand-side, it is
a store, else it is a load.
\end{itemize}
}

\frame[containsverbatim]{\frametitle{{\bf Aa} Statements}
\begin{itemize}
\item {\em Atomic} statements.
\item {\em non-Atomic} statements.
\end{itemize}
}

\frame[containsverbatim]{\frametitle{Atomic {\bf Aa} Statements}
\begin{itemize}
\item {\em Simple} statements.
\item {\em Block} statements.
\end{itemize}
}

\frame[containsverbatim]{\frametitle{{\bf Aa} Atomic Simple Statements}
\begin{itemize}
\item Assignment statements:
\begin{verbatim}
target-expression := source_expression
e.g.
a := (b + c)
\end{verbatim}
\item Call statements:
\begin{verbatim}
$call fpadd32 (A B) (C)
\end{verbatim}
\end{itemize}
}

\frame[containsverbatim]{\frametitle{{\bf Aa} Atomic Block Statements}
\begin{itemize}
\item Series-block statements.
\item Parallel-block statements.
\item Branch-block statements.
\item Fork-block statements.
\end{itemize}
}

\frame[containsverbatim]{\frametitle{{\bf Aa} Series Block Statements}
\begin{verbatim}
$seriesblock [SB] {
   $storage a: $uint<32>
   a := (b + c)
   d := (a + e)
} (d => D)
\end{verbatim}

Control-flow is sequential: statements are executed in order, token
leaves statement after last statement finishes.  
\vspace{0.5cm}
A module body is also a series-block.
}

\frame[containsverbatim]{\frametitle{{\bf Aa} Parallel Block Statements}
\begin{verbatim}
$parallelblock [PB] {
   a := (b + c)
   p := (q + r)
}
\end{verbatim}
Control-flow: both statements started in parallel,  token leaves
statement after both statements have finished.
}

\frame[containsverbatim]{\frametitle{{\bf Aa} Branch Block Statements}
\begin{verbatim}
$branchblock [BB] {
   $merge $entry loopback
       $phi I := ($bitcast($uint<32>) 0) $on $entry
                    NI $on loopback
   $endmerge
   a[I] := (b[I] + c[I])
   NI := (I+1)
   $if (NI < 16) $then 
      $place [loopback] 
   $endif
}
\end{verbatim}
Control-flow:  sequential, but control flow is altered by
merge, place, if and switch statements.
}

\frame[containsverbatim]{\frametitle{{\bf Aa} Fork Block Statements}
\begin{verbatim}
$forkblock [FB] {
   $seriesblock [S1] { ... }
   $seriesblock [S2] { ... }
  
   $join S1 S2 $fork
      $seriesblock [S3] { ... }
      $seriesblock [S4] { ... }
   $endjoin
 
   $join S3 S4 $endjoin
}
\end{verbatim}
Control-flow:  all statements will start in parallel, join-forks
will trigger new statements etc.  Token exits block when all statements
finish.
}

\frame[containsverbatim]{\frametitle{{\bf Aa} Do-While Statements}
These are not atomic, and can occur only inside a branch-block.
\begin{verbatim}
$do
   $merge $entry $loopback
        $phi I := ($bitcast ($uint<32>) 0) 
                      $on $entry
                   NI $on $loopback
   $endmerge
   a[I] := (b[I] + c[I])
   NI   := (I+1)
$while (I < 16)
\end{verbatim}
Control-flow:  sequential, controlled by the condition check.  The
places \$entry and \$loopback are implicitly defined.  When control
enters the do-while, the token gets placed in \$entry and when
control loops-back from the condition check, the token gets placed
in \$loopback.
}


\frame[containsverbatim]{\frametitle{Getting to hardware starting from an {\bf Aa} Program}
}


\end{document}
