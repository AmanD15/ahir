\documentclass[12pt,a4paper]{article}
\usepackage{graphicx}
\usepackage{times}
\usepackage{amsfonts}
\usepackage[margin=0.75in]{geometry}

\title{On the use of petri nets for modelling high-level programs}

\author{Sameer D. Sahasrabuddhe \\ Kavi Arya \\ Madhav P. Desai}

\date{}

\newtheorem{definition}{Definition}

\newtheorem{srule}{Rule}

\begin{document}

\maketitle

\begin{abstract}

  Petri nets provide a powerful mechanism for the modelling of
  concurrent systems. However, because of their generality, it is
  often difficult to prove properties about petri net models unless
  some restrictions are placed on their construction. We consider the
  use of petri nets to model algorithms described in high level
  programming languages. In particular, we introduce a class of live
  and safe petri nets (which we term as {\em Type-2} petri nets) for
  this purpose. This class is defined by a set of production rules
  which ensure liveness and safety of the resulting petri net.
  
  Further, we show that Type-2 petri nets support an efficient method
  for concurrency analysis that uses a labelling scheme which is
  analogous to a symbolic simulation of the petri net. Thus, it is
  easy to determine the sets of events (or transitions) which satisfy
  ordering relationships (two transitions are said to be ordered if it
  is impossible for both of them to fire simultaneously). This is
  useful in identifying possibilities for conflict free resource
  sharing when implementing the programs being modelled.

  Despite the restrictions imposed by the production rules, the Type-2
  class of petri nets is powerful enough to represent programs written
  in traditional imperative languages such as C/C++ and also in
  synchronous languages such as Esterel. It has been used as a
  transition step in translating imperative programs to hardware
  descriptions\cite{ahir-thesis}. The resulting high-level synthesis
  flow uses the labelling scheme to identify opportunities for sharing
  hardware resources in the absence of timing information.

\end{abstract}

\section{Introduction}
\label{sec:introduction}

Petri nets provide a powerful formalism that has been used to model
and analyse the behaviour of a large class of systems. We consider the
use of petri nets to model the behaviour of algorithms described in
high-level programming languages in order to derive efficient
implementations. We use the petri net to capture the control flow in
the program which is coupled with a data-path (a set of computational
resources) in order to implement the intended behaviour. The petri net
model can be analysed to unveil properties that are important when
deriving an implementation, such as safety and boundedness. In
addition, the analysis can also provide information for efficient
utilisation of resources in the data-path.

However, the complexity of analysing a petri net in general quickly
increases with its size. We adopt the strategy of reducing this
complexity by restricting the structure of the petri net itself. We
propose a class of petri nets (that we call ``Type-2'') that is live
and safe, and can be easily analysed to determine concurrency
relationships between different computations in the program.

The structure of the Type-2 petri net is well suited for representing
the control flow observed in high-level programs --- in particular,
the interaction between concurrency and choice (branches). Threads
that are started within a branch must synchronise before exiting the
branch, and conversely, the targets of a branch that occurs in a
thread must be restricted to that thread. A Type-2 petri net is built
using a set of construction rules that restrict the structure of the
petri net to ensure this behaviour. The same restrictions enable an
efficient labelling scheme that can be used to analyse the concurrency
of operations in the program. Note that the behaviour allowed by a
Type-2 petri net is a generalised version of a series-parallel
net\cite{something}, as can be seen in Section~\cite{sec:somewhere}.

\section{Definitions}
\label{sec:defintions}

A petri net is a directed bipartite graph with two kinds of nodes ---
\textit{places} and \textit{transitions}. A \textit{marking} is a
function that assigns a non-negative number to each place, indicating
the number of tokens in that place.

\begin{eqnarray*}
  N & = & (P, T, E, M)\\
  \textrm{where } P & = & \textrm{finite set of places}\\
  T & = & \textrm{finite set of transitions}\\
  E & \subseteq & (P \times T) \cup (T \times P)\\
  M & : & P \rightarrow \mathbb{N}
\end{eqnarray*}

The above defintion allows at most one edge from a place (transition)
to a transition (place). A \textit{simple place (transition)} is a
place (transition) with one incoming edge and one outgoing place. A
\textit{branch (fork)} is a place (transition) with one incoming edge
and multiple outgoing edges, while a \textit{merge (join)} is a place
(transition) with multiple incoming edges and one outgoing edge.

\section{Type-1 petri nets}
\label{sec:type-1}

We first define the class of Type-1 petri nets, which are constructed
by a set of construction rules combined with the most basic Type-1
petri net. The class of Type-2 petri net is a subclass that is defined
by an additional construction rule introduced in
Section~\ref{sec:type-2}.

\begin{definition}
  A \emph{simple cycle} is the petri net $N = (P, T, E, M_0)$ where $P
  = \{p\}$, $T =\{t\}$, $E = \{(p,t), (t,p)\}$ and the
  initial marking $M_0(p) = 1$.
\end{definition}

Clearly, a simple cycle is live and safe. It is defined as the
simplest type-1 petri net, and any petri net obtained by repeatedly
applying one or more construction rules to the simple cycle is also a
Type-1 petri net. Each rule replaces a simple place or transition in
the original petri net with a subgraph called a ``standard region''.
There are three types of standard regions, defined below. Places
occuring in the standard regions are \textbf{not marked} in the
initial marking.

\begin{description}
\item [Series region:] A \emph{p-type series region} (t-type series
  region) is a petri net consiting of a sequence of simple transitions
  and places, where the first and the last elements are both places
  (transitions) with only one edge within the region. The first and
  the last elements of the series region are called the \emph{entry}
  and the \emph{exit} of the region respectively.

\item [Branch region:] A \emph{branch region} is a conected petri net
  where every transition is simple, while every place is either a
  branch or a merge. No place in the branch region has an edge
  incident on an element outside the region, except the \emph{entry}
  which has one incoming edge from outside the region and the
  \emph{exit} which has one outgoing edge outside the region.

\item [Fork region:] A \emph{fork region} is a connected petri net
  where every place is simple, while every transition is either a fork
  or a join. No transition in a fork region has an edge incident on an
  element outside the region, execep the \emph{entry} and \emph{exit}
  transitions, analogous to the entry and exit of a branch region. In
  addition, a fork region is required to be a DAG.
\end{description}

The construction rules for obtaining a  Type-1 petri net can now be
stated as follows:

\setcounter{srule}{-1}

\begin{srule}
  The simple cycle is a Type-1 petri net.
\end{srule}

\begin{srule}
  The petri net obtained by replacing an unmarked simple place
  (transition) in a Type-1 petri net by a p-type series region (t-type
  series region) is a also a Type-1 petri net.
\end{srule}

\begin{srule}
  The petri net obtained by replacing an unmarked simple place in a
  Type-1 petri net by a branch region is also a Type-1 petri net.
\end{srule}

\begin{srule}
  The petri net obtained by replacing a simple transition in a Type-1
  petri net by a fork region is also a Type-1 petri net.
\end{srule}

The above construction rules ensure that the place occuring in the
original simple cycle is the only marked place and it is never
replaced during the construction of the Type-1 petri net.

\section{Liveness and Safety in a Type-1 petri net}
\label{sec:live-safe}

We prove that a Type-1 petri net is live and safe by induction over
the structure of the petri net. In particular, we show that the
replacement of a simple transition or place in a Type-1 petri net by
applying any of the construction rules does not affect the liveness
and safety of the rest of the petri net.

\subsection{Simplification}
\label{sec:simplification-live-safe}

Consider a simple place $p$ in a live and safe petri net $N =
(P,T,E,M)$. Let $u,v$ be the predecessor and successor transitions of
$p$ in $N$, i.e, $u,v \in T$ and $(u,p), (p,v)\in E$. Since $N$ is
live and safe, it follows that when $u$ fires, $v$ must fire before
$u$ may fire again (safety), and $v$ must eventually fire when $u$
fires (liveness). Thus, any firing sequence of the petri net $N$ that
includes the firing of $u$ (and consequently also $v$) effectively
puts a token in $p$ and then eventually removes it. Thus, the petri
net $N' = N - \{p,u,v\}$ can be modelled as an initially marked simple
place $p_0$ that forms a cycle with $u$, $v$ and $p$. Similarly,
relative to a simple transition $t$, the rest of the petri net can be
modelled as an initially marked simple place $p_0$ that forms a simple
cycle with $t$. Both replacements are valid if and only if the
original petri net is live and safe; the resulting simple cycle is
also live and safe. We use this simplification to examine the effect
of applying a construction rule on a Type-1 petri net:

\begin{enumerate}
\item When a rule replaces a simple place $p$, the rest of the petri
  net is reduced to a simple marked place $p_0$ connected to $p$ by
  simple transitions.
\item When a rule replaces a simple transition $t$, the rest of the
  petri net is first reduced to a single marked place $p_0$.
\end{enumerate}

Thus, in each of the following subsections, we model an arbitrary
Type-1 petri-net as a cycle consisting of a marked place $p_0$
connected to the candidate simple place or transition through a
suitable number of additional simple transitions. We refer to this
cycle as the \emph{base petri net} in the rest of this section.

\subsection{p-type Series Region}
\label{sec:series-live-and-safe}

When a simple place (transition) in the base petri net is replaced by
p-type series region (t-type series region), the result is simply a
larger cycle. The resulting petri net is live and safe, since it is a
strongly connected \emph{state machine} with exactly one marked place.

\subsection{Branch Region}
\label{sec:branch-live-and-safe}

When a place in the base petri net is replaced by a branch region, the
resulting petri net is a strongly connected \emph{state machine} which
is that has exactly one marked place. Hence it is live and
safe\ref{murata-tutorial}.

\subsection{Fork Region}
\label{sec:fork-live-and-safe}

When a transition in the base petri net is replaced by a fork region,
the resulting petri net is a \emph{marked graph} such that every cycle
in the marked graph pass through the single marked place $p_0$. Thus
every cycle in the petri net has a marked place (hence it is live) and
every place is on a cycle with exactly one marked place (hence it is
safe).

\end{document}

