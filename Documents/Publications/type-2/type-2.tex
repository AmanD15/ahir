\documentclass[12pt,a4paper]{article}
\usepackage{graphicx}
\usepackage{times}
\usepackage[margin=0.75in]{geometry}

\title{On the use of petri-nets for modeling concurrent programs}

\author{Sameer D. Sahasrabuddhe \\ Kavi Arya \\ Madhav P. Desai}

\date{}

\begin{document}

\maketitle

\begin{abstract}
  \large

  Petri-nets provide a powerful mechanism for the modeling of
  concurrent systems.   However, because of their generality, 
  it is often difficulty to prove properties about petri-net
  models unless some restrictions are placed on their construction.
  We consider the use of petri-nets to model algorithms described
  in high level programming languages.  In particular, we introduce
  a class of live and safe petri-nets (which we term as {\em Type-2}
  petri-nets) for this purpose.   This class is defined by a
  set of production rules which ensure liveness and safety of
  the resulting petri-net.  
  Further, we show that concurrency
  analysis of Type-2 petri-nets can be carried out very efficiently
  using a labelling scheme which is analogous to a symbolic simulation
  of the petri-net.
  Thus, it is easy to determine the sets of events (or transitions)
  which satisfy ordering relationships (two transitions are said to
  be ordered if it is impossible for both of them to fire simultaneously).  
  Such information is useful in identifying possibilities for conflict 
  free resource sharing in the implementations of the programs being
  modeled.
  Despite the restrictions imposed by the production rules, the Type-2
  class of petri-nets is powerful enough to represent programs
  written in traditional imperative languages such as C/C++ as well in synchronous
  languages such as Esterel. It has been used as a transition step in translating
  imperative programs to hardware descriptions\cite{ahir-thesis}. The
  resulting high-level synthesis flow uses the labelling scheme to
  optimise the generated hardware by identifying opportunities for
  sharing hardware resources in the absence of timing information.

\end{abstract}

\end{document}
