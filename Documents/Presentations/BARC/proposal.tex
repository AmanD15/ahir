\documentclass[12pt]{article}

\usepackage{geometry}
\geometry{verbose,a4paper,tmargin=1in,bmargin=1in,lmargin=1in,rmargin=1in}

\usepackage{graphicx}
\usepackage{multirow}

\newenvironment{packed_enum}{
\begin{enumerate}
  \setlength{\itemsep}{1pt}
  \setlength{\parskip}{0pt}
  \setlength{\parsep}{0pt}
  \setlength{\partopsep}{0pt}
  \setlength{\topsep}{0pt}
}{\end{enumerate}}

\begin{document}

\newcounter{srno}
\newcommand{\setsr}[1]{\setcounter{srno}{#1}}

\newcounter{srinc}
\newcommand{\incten}[0]{\setcounter{srinc}{10}}
\newcommand{\incone}[0]{\setcounter{srinc}{1}}

\newcommand{\srno}[2][\arabic{srno}]{\noindent\paragraph{#1. #2}\addtocounter{srno}{\value{srinc}}}

\incone
\setsr{100}

\newcommand{\mytitle}{Development of a high-level hardware synthesis
  tool-chain}

\begin{titlepage}
\begin{large}
\begin{center}
\vspace{0.5cm}
Project Proposal\\
\vspace{0.5cm}
for\\
\vspace{0.5cm}
{\Large \bf \mytitle}\\
\vspace{2.5cm}
\begin{tabular}{l}
\hline
{\bf Principal Investigator}\\
\hline
Prof. Kavi Arya\\
Centre for Formal Design and Verification of Software, and\\
Department of Computer Science and Engineering,\\
Indian Institute of Technology - Bombay,\\
Powai, Mumbai - 400076\\
\\
kavi@cse.iitb.ac.in\\
\hline
\end{tabular}
\end{center}
\end{large}
\end{titlepage}


\begin{center}
  \bf \Large Section-A\\
  \large Part I\\
  \vspace{2ex}
  Project Overview\\
\end{center}

\srno{Advisory Committee Code Number:} 34 (ATC)

\srno{Title:}
\begin{center}
  \large\mytitle
\end{center}

\srno{Keywords: } high-level synthesis, hardware compilation, EDA

\srno{Project Summary:}

\paragraph{} High-level synthesis is the process of compiling
high-level programs to hardware circuits. This reduces the cost of
hardware design by providing two advantages. Firstly, it lowers the
level of expertise required so that even programmers can generate
hardware designs, and secondly, it significantly reduces the time and
effort required to create efficient implementations. High-level
synthesis is also relevant to safety-critical applications since it
guarantees provably correct implementations of the input
specification. The resulting hardware does not have to be verified;
only the behaviour of the input program has to be functionally
verified.

Current work at IIT Bombay has established a scalable automated
high-level synthesis flow that guarantees correctness, called AHIR. It
has been tested on diverse examples such as cryptography, linear
algebra, data structures and signal processing. The flow takes
advantage of established compiler techniques to infer what the user
meant when they wrote the program, and produce the most suitable
hardware for it.

The proposed project aims to make the ongoing research at IIT Bombay
available to BARC in the form of a synthesis tool-chain based on AHIR.
The tool-chain will also be demonstrated on a system currently being
developed at BARC --- a hardware implementation of the TCP/IP network
protocol stack. Such a system is being designed at BARC using the
commercial high-level synthesis product called Handel-C. A second
implementation will be created under the proposed project, using AHIR
to synthesise a C implementation of TCP/IP to hardware. This will
provide the basis for a practical evaluation of the tool-chain.

The proposed project is a starting point for long-term development
work on the use of high-level synthesis at BARC. Access to such an
indigenous technology is of strategic importance to BARC in terms of
self-reliance. The users at BARC will have direct access to the
research group working on AHIR at IIT Bombay. This will allow BARC to
keep pace with further developments in high-level synthesis including
optimisations and the ability to target emerging hardware platforms.

\newpage

\srno{Principal Investigator}
\paragraph{}
\begin{tabular}{l}
  Prof. Kavi Arya\\
  \\
  Centre for Formal Design \& Verification\\
  of Software, and\\
  Dept. of Computer Science and Engg.,\\
  Indian Institute of Technology - Bombay,\\
  Powai, Mumbai - 400076\\
  \\
  kavi@cse.iitb.ac.in\\
\end{tabular}

\srno{Co-Investigator}
\paragraph{}
\begin{tabular}{l}
  Prof. Madhav P. Desai\\
  \\
  Dept. of Electrical Engineering,\\
  Indian Institute of Technology - Bombay,\\
  Powai, Mumbai - 400076\\
  \\
  madhav@ee.iitb.ac.in\\
\end{tabular}

\srno{Principal Collaborator}
\paragraph{}

\setsr{109}

\newpage

\srno{Budget Amounts in Rupees (Lakhs): }

\begin{center}
\begin{tabular}{|c|c|c|c|c|c|c|}
\hline
\multicolumn{2}{|c|}{Particulars} & Year I & Year II & Year III & Total \\
\hline
\hline
Equipment &  & 7.25 & -- & -- & 7.25 \\
Salaries & JRF & 1.38 & 1.38 & 1.56 & 4.32 \\
 & JRF & -- & 1.38 & 1.56 & 2.94 \\
 & RA-III & 4.20 & 4.20 & 4.80 & 13.20 \\
Consumables &  & 0.25 & 0.25 & 0.25 & 0.75 \\
Travel Allowance & PI & 1.05 & 1.05 & 1.05 & 3.15 \\
 & PC & 0.20 & 0.20 & 0.20 & 0.60 \\
Technical Assistance &  & 0.15 & 0.15 & 0.15 & 0.45 \\
Contingencies &  & 0.15 & 0.15 & 0.15 & 0.45 \\
\hline
Total &  & 14.63 & 8.76 & 9.72 & 33.11 \\
\hline 
Overheads &  & 2.19 & 1.31 & 1.46 & 4.97\\
\hline
Total &  & 16.82 & 10.07 & 11.18 & 38.08\\
\hline
\end{tabular}
\end{center}

\srno{Equipment Cost:}

\begin{center}
\begin{tabular}{|c|c|c|c|}
\multicolumn{4}{l}{\bf Amounts in Rupees (Lakhs)}\\
\hline
Item & Quantity & Rate & Amount \\
\hline
Desktop PC & 3 & 0.50 & 1.50 \\
External hard drives & 2 & 0.05 & 0.10 \\
Laptop & 1 & 0.65 & 0.65 \\
FPGA board & 2 & 1.00 & 2.00 \\
\hline
Total & & & 4.25 \\
\hline
\end{tabular}
\end{center}

\newpage

\begin{center} {\bf \large Part-II\\
\vspace{2ex}
Project Objectives, Research Plan and Deliverables}
\end{center}

\setsr{200}
\incten

\srno{List of Objectives}

The overall purpose of the project is to develop a high-level
synthesis software suite to be used at BARC, based on the research
being done at IIT Bombay. In particular there are two objectives:

\begin{enumerate}
\item Deliver a software tool-chain based on AHIR that provides the
  following facilities:
  \begin{enumerate}
  \item Compile C programs to synthesisable RTL descriptions.
  \item Run and evaluate the generated hardware on an FPGA board.
  \item Test and debug the behaviour of the FPGA implementation with
    respect to the input C programs.
  \end{enumerate}

\item Evaluate the effectiveness of the high-level synthesis flow by
  automatically generating a hardware implementation of the TCP/IP
  network protocol stack, starting from a behavioural description
  written in C.
\end{enumerate}

\srno{Yearly Research Plan and Deliverables}

\paragraph{} A high-level synthesis tool-chain based on AHIR was
implemented as part of of ongoing research at IIT Bombay on high-level
synthesis. The tool-chain consists of a set of tools that can
translate a subset of the C programming language to hardware modules.
It has been tried on a number of examples that represent various
applications such as cryptography, digital signal processing,
high-performance computing and high-level data structures. Work done
under this project will result in a complete software bundle being
delivered to BARC for regular use. This includes the implementation of
on-going research work at IIT Bombay aimed at enabling the compilation
of very large systems specified as C programs.

We propose to use a C implementation of the TCP/IP network protocol
stack as a working example during development. This will serve to
demonstrate the effectiveness of the synthesis tool-chain. Insights
gained during this exercise will also provide data for the on-going
research at IIT Bombay.

\subsubsection*{Developing the tool-chain}

The existing implementation at IIT Bombay has been used successfully
to demonstrate the potential gains of a high-level synthesis flow
based on AHIR. The work under the proposed project will be aimed at
realising these potential benefits and also packaging the prototype as
a complete end-to-end solution. In particular, the work will be
concentrated on following areas of development:

\begin{enumerate}
\item Support for a minimum set of features including the following:
  \begin{enumerate}
  \item Complete support for the ISO C language standard.
  \item Input/Output ports.
  \item Separate compilation and pre-compiled libraries.
  \end{enumerate}
\item Improvements in the hardware produced, including the following:
  \begin{enumerate}
  \item Improvements in the library of building blocks --- efficient
    operators, multicycle operators, pipelined operators etc.
  \item Efficient utilisation of hardware resources.
  \item Improvements in throughput by automatically recognising and
    eliminating unnecessary sequential delays when generating
    hardware.
  \item Support for sharing of pipelined multicycle operators.
  \end{enumerate}
\item Support for the design and implementation of large systems by
  separately designing various modules and automatically composing
  them into a single coherent system.
\end{enumerate}

The development work on these areas will involve a synergy between the
proposed project and the ongoing research work on AHIR at IIT Bombay.
Of the three areas of thrust highlighted here, the first two are
largely concerned with improving the existing implementation of the
AHIR tool-chain. The third area deals with extending the current
implementation to include new features being developed at IIT Bombay.
The current tool-chain takes a single C program and generates a
hardware implementation for it. But in order to support a large
system, the tool-chain must address the following issues:

\begin{enumerate}
\item Concurrent execution of modules in a system.
\item Efficient communication channels between the modules.
\item A memory subsystem for external storage of data, that can keep
  up with the demands of the system modules.
\item Facility to use a high-level description of the system
  architecture. The entire system will be automatically generated from
  this high-level description by combining the individual modules.
\end{enumerate}

Research work in this area being done at IIT Bombay will greatly
benefit from experience in using the existing AHIR tool-chain in
implementing a non-trivial system in hardware. The proposed project
will assist in this respect through the development of a TCP/IP
network protocol stack in hardware for use at BARC.

\subsubsection*{TCP/IP implementation}

The development of the TCP/IP implementation will begin in parallel
with the work on the AHIR tool-chain. Initial work will largely focus
on characterising the major components of such a system, and building
an early version using the existing AHIR implementation. Each
component will be implemented as a C program and compiled to hardware
using AHIR. The components will be put together manually to produce
the first prototype of the entire system.

The first manual effort is important in terms of highlighting the
automatic functionality expected from the AHIR tool-chain, that allows
a user to describe a large system as a collection of C programs, and
expect to automatically generated an efficient hardware
implementation. The result will be a complete high-level specification
of the TCP stack, that can be automatically compiled to competitive
hardware using AHIR.

\subsubsection*{Yearly Deliverables}

  \renewcommand\arraystretch{1.2}
\begin{tabular}{|c|p{0.45\textwidth}|p{0.45\textwidth}|}
  \hline
  Year & Activity & Deliverable\\
  \hline
  \hline
  I & Identifying and implementing ISO C features
  that are not  supported in the current implementation. &
  \multirow{4}{0.45\textwidth}{Complete    support for the C language
    in the AHIR tool-chain.}\\
  \cline{2-2}
  & Implementing I/O ports. & \\
  \cline{2-3}
  & Implementing multicycle and pipelined operators. & A library of
  efficient hardware operators to be used by the generated circuit.\\
  \cline{2-3}
  & Improving parallelisation of operators and memory accesses. &
  Improved throughput in the generated circuits. \\
  \cline{2-3}
  & \multirow{3}{0.45\textwidth}{Designing the overall architecture of
    the TCP/IP hardware    implementation.} & A preliminary overview
  of the system architecture,  along with expected performance and
  hardware costs. \\
  \hline
  \hline
  II & Implementing hardware optimisations including sharing of
  pipelined operators, sharing of registers and system-level resource
  reuse. & \multirow{4}{0.45\textwidth}{Hardware optimisation module
    in the AHIR tool-chain.} \\
  \cline{2-3}
  & Implementing separate compilation and    library linking. &
  Support  for pre-compiled libraries in the  AHIR  tool-chain.\\ 
  \cline{2-3}
  & \multirow{5}{0.45\textwidth}{Implementing the components in the
    TCP/IP system as separate     programs and manually combining them
    for a complete implementation.}   & A preliminary implementation
  of the TCP/IP system.\\
  \cline{3-3}
  & & A characterisation of the meta-level information required for
  automatic generation of entire systems. \\
  \hline
  \hline
  III & Implementing a system-level synthesis tool
  that combines  independent components using a high-level
  architectural description  to produce a complete system. & A
  complete high-level synthesis  tool-chain for generating complex
  systems from large programs. \\ 
  \cline{2-3}
  & Automatically generating a hardware of the TCP/IP network protocol
  stack using the AHIR tool-chain. & A complete high-level description
  of the TCP/IP stack along with a  working implementation on an FPGA.\\
  \hline
\end{tabular}

\newpage

\begin{center} {\bf \large Part-III\\
\vspace{2ex}
Budget Estimates}
\end{center}

\setsr{290}
\incten

\srno{Details of Budget Requirements --- Amounts in Rupees (Lakhs)}

\begin{center}
\begin{tabular}{|c|c|c|c|c|c|c|}
\hline
\multicolumn{2}{|c|}{Particulars} & Year I & Year II & Year III & Total \\
\hline
\hline
Equipment &  & 7.25 & -- & -- & 7.25 \\
Salaries & JRF & 1.38 & 1.38 & 1.56 & 4.32 \\
 & JRF & -- & 1.38 & 1.56 & 2.94 \\
 & RA-III & 4.20 & 4.20 & 4.80 & 13.20 \\
Consumables &  & 0.25 & 0.25 & 0.25 & 0.75 \\
Travel Allowance & PI & 1.05 & 1.05 & 1.05 & 3.15 \\
 & PC & 0.20 & 0.20 & 0.20 & 0.60 \\
Technical Assistance &  & 0.15 & 0.15 & 0.15 & 0.45 \\
Contingencies &  & 0.15 & 0.15 & 0.15 & 0.45 \\
\hline
Total &  & 14.63 & 8.76 & 9.72 & 33.11 \\
\hline 
Overheads &  & 2.19 & 1.31 & 1.46 & 4.97\\
\hline
Total &  & 16.82 & 10.07 & 11.18 & 38.08\\
\hline
\end{tabular}
\end{center}

\srno{Equipment}

\paragraph{} The PI proposes to procure three desktop PCs along with
two FPGA boards that will be used as the accelerator platform. The PI
also proposes to provide a laptop to research staff at RA-III level
for purposes of presentation of the technical work, progress reports
and documentation during the duration of the project. In addition,
this laptop will also carry the actual code of the tool for
development, debugging and demonstration purposes. This will be
important to ensure that there are no impediments to progress of the
work even when the key persons are travelling and may not be present
at the same physical location. Since the development work involves a
large amount of data, the PI proposes to purchase external hard drives
to be used with the desktop PC and the laptop. The budget requirements
for this equipment are listed below:

% \begin{enumerate}
% \item \textbf{Desktop PC:} Intel Core 2 Duo processor, 4GB DDR2 RAM,
%   2$\times$250GB SATA HDD, 19" TFT LCD monitor, DVD writer, keyboard,
%   mouse.
% \item \textbf{Laptop:} Intel Core 2 Duo processor, 2GB DDR2 RAM, 250GB
%   SATA HDD, 14" WXGA LCD, 10/100 Ethernet, 802.11g, Bluetooth.
% \item \textbf{FPGA board:} 
% \end{enumerate}

\begin{center}
\begin{tabular}{|c|c|c|c|}
\multicolumn{4}{l}{\bf Amounts in Rupees (Lakhs)}\\
\hline
Item & Quantity & Rate & Amount \\
\hline
Desktop PC & 3 & 0.50 & 1.50 \\
External hard drives & 2 & 0.05 & 0.10 \\
Laptop & 1 & 0.65 & 0.65 \\
FPGA board & 1 & 5.00 & 5.00 \\
\hline
Total & & & 7.25 \\
\hline
\end{tabular}
\end{center}

\newpage
\srno{Staff Salary --- Amounts in Rupees}

\paragraph{}The PI proposes to recruit two research staff for
undertaking the development work for the first year, and three
research staff for the second and third year. One position will be at
the level of RA-III for all three years, for a person with a PhD in
CSE/EE. One position at the level of JRF will be created for the first
year, with an additional position at the level of JRF for the second
and third year. The JRF positions will be open for persons with a
BTech in CSE/EE. The detailed budget estimate for staff salaries are
given below.

\begin{center}
\begin{tabular}{|c|c|c|c|c|}
\multicolumn{5}{l}{\bf Amounts in Rupees (Lakhs)}\\
\hline
Staff & Year I & Year II & Year III & Total \\
\hline
JRF & 1.38 & 1.38 & 1.56 & 4.32 \\
 & @Rs.11,500 p.m. & @Rs.11,500 p.m. & @Rs.13,000 p.m. & \\
\hline
JRF & -- & 1.38 & 1.56 & 2.94 \\
 & & @Rs.11,500 p.m. & @Rs.13,000 p.m. & \\
\hline
RA-III & 4.20 & 4.20 & 4.80 & 13.20 \\
 & @Rs.35,000 p.m. & @Rs.35,000 p.m. & @Rs.40,000 p.m. & \\
\hline
Total & 5.58 & 6.96 & 7.92 & 20.46 \\
\hline
\multicolumn{5}{l}{{\bf Note:} The HRA paid is 30\% of the consolidated salary.}
\end{tabular}
\end{center}

\setsr{330}

\srno{Consumables}

\paragraph{} The budgetary requirement of consumables like printer
cartridges, paper, backup CDs and other stationaries, is given below.

\begin{center}
\begin{tabular}{|c|c|c|c|c|}
\multicolumn{5}{l}{\bf Amounts in Rupees (Lakhs)}\\
\hline
Item & Year I & Year II & Year III & Total \\
\hline
Papers, Cartridges, CDs & 0.25 & 0.25 & 0.25 & 0.75 \\
and other consumables & & & & \\
\hline
Total & 0.25 & 0.25 & 0.25 & 0.75 \\
\hline
\end{tabular}
\end{center}

\newpage

\srno{Travel}

\paragraph{} It is expected that the research and development work
undertaken as part of this project will lead to at least two to three
international publications. This will help in getting the work peer
reviewed by the research community working in related areas of
verification. This may require travelling within India and also abroad
to present the research results in peer reviewed conferences. The
travel budget shown below also includes the budget required for travel
by the PC to PI's institute and by the PI to PC's institute for
regular meetings.

\begin{center}
\begin{tabular}{|c|c|c|c|c|}
\multicolumn{5}{l}{\bf Amounts in Rupees (Lakhs)}\\
\hline
Item & \multicolumn{4}{c|}{Funds required} \\
\cline{2-5}
 & Year I & Year II & Year III & Total \\
\hline
Weekly visits by PC & 0.20 & 0.20 & 0.20 & 0.60 \\
to PI's premises & & & & \\
Monthly visits by PI & 0.05 & 0.05 & 0.05 & 0.15 \\
to PC's premises & & & & \\
Attending national \& & 1.00 & 1.00 & 1.00 & 3.00 \\
international conferences & & & & \\
\hline
Total & 1.25 & 1.25 & 1.25 & 3.75 \\
\hline
\end{tabular}
\end{center}

% \srno{Contingency}

% \srno{Overheads}

% \srno{Grand Total}

% \begin{center}
%   \large
%   \vspace{2ex}
%   BUDGET DETAILS
%   \vspace{2ex}
% \end{center}

% \srno[310]{Details of the budget for equipment to be procured by the
%   PI (Amount in Rupees):}

% \srno[320]{Details of staff salary (Amount in Rupees):}

% \srno[340]{Details of budget for consumables to be procured by the PI
%   (Amount in Rupees):}

% \setsr{350}
% \incone

% \srno{Details of travel:}
% \srno{Proposed number of visits of PC/DC to PI's institute}
% \srno{Duration of stay during each visit (No. of days)}
% \srno{Total funds required}
% \srno{Proposed number of visits of PI to PC/DC's institute}
% \srno{Duration of stay during each visit (No. of days)}
% \srno{Total funds required}
% \srno{Funds required by PI for travel to attend conferences within India.}
% \srno{Funds for Other visits}


% \begin{center}
%   \large
%   \vspace{2ex}
%   BUDGET JUSTIFICATIONS
%   \vspace{2ex}
% \end{center}

\newpage

\begin{center}
  \bf \Large Section-B\\
  \vspace{2ex}
\end{center}

\incone
\setsr{499}
\srno{Technical Information}

\section{Introduction}

Given an algorithm, one can either implement it on a microprocessor,
or design hardware with an equivalent behaviour. Although compiling a
program to a microprocessor is convenient, a hardware implementation
has a number of advantages:

\begin{enumerate}
  \item Higher performance.
  \item Decreased power dissipation.
  \item Reduced size.
\end{enumerate}

But hardware platforms continue to evolve, making it possible to
integrate ever larger systems onto a single device. As the size
increases, it becomes more and more difficulty to cope with the
complexity of the design. In addition, it is important that the design
be independent of the underlying hardware, so that the target platform
may be changed without affecting the design itself. This can be only
be addressed by making effective use of {\em high-level synthesis},
the process of translating high-level programs into hardware.

The use of high-level programs as the starting point makes hardware
design accessible to a very large set of users. If the compiler is
guaranteed to produce a correct implementation of the input
specification, the resulting hardware does not have to be verified. 
Only the behaviour of the input specification has to be verified,
which can be done using existing software verification practices.

A competitive high-level synthesis process must preserve common
practices in software programming while providing a verifiable and
optimisable path to a hardware implementation. This requires a
compiler flow that can generate efficient circuits from complex
high-level programs. Such a compiler flow must have the following
features:

\begin{enumerate}
  \item It should be independent of the programming language
        used.
  \item It should guarantee a correct implementation of the
        specified behaviour.
  \item It should support optimisations that can scale to very
        large systems.
  \item It should be able to target new hardware platforms as
        they emerge.
\end{enumerate}

AHIR is an attempt at implementing such a high-level synthesis flow. 
AHIR uses an intermediate representation that makes it possible to
decouple high-level issues encountered when writing programs, from
low-level issues in the hardware implementation. An AHIR specification
is factorised into three components: control-flow, data-flow and
storage. The three components can be analysed and transformed
separately without affecting each other, enabling optimisations that
can scale with the size of the circuit. At the same time, AHIR
captures sufficient information from the input program, so that it can
be routinely mapped to a variety of hardware platforms.

\section{AHIR}

AHIR is a compact graph-based hardware representation that provides a
delay-independent specification of the circuit. A specification in
AHIR can be routinely translated to hardware. Features such as
parallelism and resource sharing may be available in the input
language, or introduced by the compiler. In either case, the
intermediate representation is rich enough to describe them.

\begin{figure}[h]
  \centering
  \includegraphics[scale=0.25]{omega.pdf}
  \caption{An AHIR system.}
  \label{fig:system}
\end{figure}

A system in AHIR consists of a number of modules, each consisting of a
control-path and a data-path. Typically, each function in an input
program is translated to a module. Function calls are implemented by
passing the call request along with the relevant arguments through an
inter-module link layer. The modules can also share data through an
external memory subsystem.

AHIR does not include any details about the memory architecture; only
a simple linear address space is assumed. Each module can specify an
arbitrary number of memory access ports. The only requirement from the
memory subsystem is that a request made on an access port must be
served eventually.

\begin{figure}[h]
\centering
\includegraphics[scale=0.3]{decouple.pdf}
\caption{An AHIR module.}
\end{figure}

A module in AHIR consists of two closely interacting components: a
control-path and a data-path. The data-path in a module is a pool of
operators connected by wires. The control-path is a petri-net that
specifies the ordering of events in the module. AHIR uses a subclass
of petri-nets called {\em Type-2 petri-nets}, which has been defined
to guarantee liveness and safety, while supporting scalable analyses.

%\begin{figure}[h]
%\centering
%\includegraphics[scale=0.35]{alpha-schema.pdf}
%\caption{Handshakes between control and data paths.}
%\end{figure}

The control-path and data-path interact through a link layer by
exchanging request-acknowledge handshakes that encapsulate delays
occurring in the implementation. AHIR specifies a set of constraints
on the delays involved --- an implementation that satisfies these
constraints is guaranteed to be correct.

\begin{figure}[h]
\centering
\includegraphics[scale=0.3]{isochron.pdf}
\caption{Delay constraints in AHIR.}
\label{fig:isochron}
\end{figure}

Figure~\ref{fig:isochron} shows a hypothetical example with associated
delays. The numbered delays $d_0$ to $d_5$ in this figure are not
individual values, but representatives of their respective class of
delays. The delays in the implementation must satisfy the following
two inequalities in order to guarantee correctness:

\[d_5 \le d_0 + d_1 + d_3\]
\[d_2 \le d_3 + d_4 + d_0\]

\subsection{Key innovations}

\begin{enumerate}
\item An orthogonal representation of the control, data and storage
      aspects of the input program, that can be routinely translated
      to a hardware implementation.
\item Support for static analysis and optimisation procedures that can
      scale with the size of the program.
\item A verifiable compilation path from algorithms to digital
      hardware, based on the intermediate representation.
\end{enumerate}

\subsection{High-level Synthesis using AHIR}

\begin{figure}[h]
  \centering
  \includegraphics[scale=0.3]{flow.pdf}
  \caption{High-level Synthesis using AHIR.}
\end{figure}

The high-level synthesis flow based on AHIR consists of two phases: a
software phase that optimises and translates a high-level program to
AHIR, and a hardware phase that optimises and translates an AHIR
specification to a hardware implementation. The result is an
end-to-end compiler flow that translates C programs to synthesisable
VHDL descriptions.

The software phase consists of existing software compilation
techniques along with a backend that generates AHIR specifications. 
The translation to AHIR uses a well-known intermediate form called a
CDFG (Control Data Flow Graph). The CDFG is first derived from the
optimised program and then translated to an AHIR specification using a
simple piece-wise construction method. The method used guarantees
correctness {\it by construction} --- the generated AHIR specification
correctly implements the behaviour specified by the CDFG.

The control, data and storage components of the AHIR specification  can
now be implemented independently. The factorisation allows us to
efficiently analyse and transform these components in order to
optimise the circuit. We demonstrate this with an optimisation
that improves resource utilisation by sharing resources in the
data-path.

The resource sharing is based on a static analysis of the control-path
that identifies pairs of operations that cannot be active at the same
time. The algorithms and supplementary data structures used by the
analysis are very simple and close to linear in complexity. 
Experimental results show that even this simple approach is quite
effective in improving the resource utilisation of the circuit.

The optimised AHIR specification is translated to a circuit
implementation by the hardware phase. The compiler is free to explore
various architectures in this phase, as long as the timing constraints
in AHIR are satisfied. The constraints are easy to satisfy in practice
by sufficiently padding the relevant circuit delays. The AHIR
specification is then translated to a synchronous circuit
implementation by directly replacing each component with an equivalent
VHDL implementation.

\subsection{Results}

\begin{figure}[h]
\centering
\includegraphics[scale=0.8]{tpg.pdf}
\caption{Comparison of an FPGA implementation with the P-4 and RTL}
\label{figure:tput-area}
\end{figure}

The current implementation has been tested on a number of examples
chosen from diverse application domains: LINPACK, Red-Black Trees,
FFT, AES block cipher and A5/1 stream cipher. One set of experiments
compares the performance of an FPGA implementation with two extremes:
equivalent programs running on a microprocessor (Intel Pentium~IV),
and hand-crafted circuits that implement the same behaviour (results
obtained from third-party literature). The metric used for comparing
performance is the ratio of the throughput in terms of jobs per second
completed by the circuit to the area occupied by it in terms of
equivalent gates. Figure~\ref{figure:tput-area} shows the performance
of the different implementations on a logarithmic scale.

\begin{figure}[h]
\centering
\includegraphics[scale=0.8]{ppw.pdf}
\caption{Comparison of an ASIC implementation with the P-4
(normalised to ``1'')}
\label{figure:ppw}
\end{figure}

A second set of experiments uses an automated VHDL-to-ASIC flow based
on commercially available tools. ASIC layouts were automatically
generated for TSMC 180nm technology using this flow. Since actual SRAM
cells were not available, the experiment uses SRAM models provided by
the CACTI project. The estimated power dissipation and area were
scaled to the 130nm technology used for the Pentium~IV. 
Figure~\ref{figure:ppw} shows the ``performance~per~watt'' delivered
by the ASIC implementation, normalised by the performance of the
Pentium~IV.

The synthesis results show that it is possible to translate complex
programs to hardware circuits using AHIR. The generated circuits are
competitive with equivalent programs running on a general-purpose
microprocessor. But the performance is less than that of hand-crafted
circuits by two orders of magnitude. This gap can be bridged by
further work in three key areas: expressing parallelism in the
high-level specification, introducing optimisations on the AHIR
specification, and using customised memory subsystems that support
parallel accesses.

\subsection{Future work}

AHIR can potentially be used as a universal design platform that
unifies {\it software compilation} (translating programs to
executables) with {\it hardware compilation} (synthesising circuits
from programs). The expressive power of AHIR is evident at two levels:
as a framework for exploring hardware architectures, and as a target
for the implementation of high-level languages. A combination of these
two aspects will result in a compiler flow that can map various
classes of high-level languages to different low-level architectures.

\subsubsection{Programming languages}

Current work in translating C programs to hardware demonstrate that
AHIR can support imperative languages, with suitable implementations
for specific features. Further research should explore the possibility
of supporting system-level languages such as SystemC and Esterel.

\subsubsection{Hardware Platforms}

AHIR has already been used for mapping C programs to FPGAs and ASICs.
But AHIR itself is not limited to only these platforms. A circuit
described in AHIR can result in a custom microprocessor, for example,
by suitably modifying the data path. AHIR can also be used to
target emerging platforms such as SoCs, arrays of functional units,
ASIC-FPGA hybrids, etc.

\subsubsection{Hardware optimisations}

The synthesis of large systems starting from complex high-level
programs requires further work towards improving performance at the
system level. The compiler must be able to generate implementations
that deliver good throughput while occupying acceptable amounts of
resources. These two parameters are usually traded-off with each other
since faster computations come at the cost of more hardware, and {\it
vice versa}. AHIR provides many opportunities for creating
transformations that improve the final circuit, such as pipelining for
improved throughput and sharing resources across modules for reduction
in area.

\subsubsection{Memory subsystem}

In addition to optimised hardware, the performance delivered by the
memory subsystem is critical to the overall performance of the system. 
High-level synthesis has to be coupled with an automated memory design
flow to generate application-specific memory subsystems that can keep
up with the hardware system.

Some work has been done by related projects at IIT Bombay towards
developing simple performance models for memory subsystems. Such
models will support an integrated memory sub-system design procedure
that can explore large areas of the design space in a feasible manner. 
For a given application, the design process may be parameterised by
the properties of the expected memory access trace.

\subsection{Publications}

\begin{enumerate}
\item ``AHIR: A Hardware Intermediate Representation for Hardware
      Generation from High-level Languages'' by Sameer D. 
      Sahasrabuddhe, Hakim Raja, Kavi Arya and Madhav P. Desai,
      presented at the 20th International Conference on VLSI Design,
      January 2007.
\item PhD Thesis entitled ``A competitive pathway from high-level
      programs to hardware specifications'' by Sameer D. Sahasrabuddhe
      (Dept. of CSE, IIT Bombay), submitted for review in October
      2008.
\end{enumerate}

% \newpage

% \begin{center} {\bf \large Part-V\\
% \vspace{5ex}
% Appendix}
% \end{center}

% \section{CV of Principal Investigator (PI)}

% \section{CV of Principal Collaborator (PC)}

\end{document}
