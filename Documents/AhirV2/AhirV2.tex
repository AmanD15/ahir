%%\begin{figure}
%%\begin{centering}
%%\centerline{\psfig{figure=OurApproach.eps,width=3.0in,height=3.0in}}
%%\caption{Our Approach to Fault Simulation using a Hardware Emulator}
%% \label{fig:OurApproach}
%%\end{centering}
%%\end{figure}

\documentclass[12pt]{article}
\usepackage{epsfig}

\title{AhirV2: from algorithms to hardware \\ An overview }
\author{Madhav Desai \\ Department of Electrical Engineering \\ Indian Institute of Technology \\
	Mumbai 400076 India}

\newcommand{\Aa}{{\bf Aa}~}
\newcommand{\vC}{{\bf vC}~}

\begin{document}
\maketitle

\section{What is AhirV2?}

AhirV2 is a set of tools which can convert a C 
description of a system to  an equivalent hardware
implementation (described in VHDL).  Using these tools,
it is possible to take an algorithmic approach to 
the design of  hardware.

The flow of transformations is illustrated in Figure \ref{fig:AhirFlow}.
\begin{figure}
\begin{centering}
\centerline{\psfig{figure=AHIRv2Flow.eps,width=3.0in,height=3.0in}}
\caption{AhirV2 flow}
 \label{fig:AhirFlow}
\end{centering}
\end{figure}

\begin{itemize}
\item Given a high-level C, we rely on
an LLVM (www.llvm.org) compatible compiler such as clang (www.clang.org)
to produce LLVM byte code.
Currently, the AhirV2 flow uses LLVM byte code as a starting point.
\item The LLVM byte-code program is compiled
to an intermediate assembly form.  AhirV2
introduces an intermediate assembly language
\Aa  which serves as a target for sequential
programming languages (such as C) as
well as for parallel programming languages.
An \Aa program consists of modules (analogous
to sub-programs in C) which can call each 
other, and can communicate through storage objects
as well as through pipes (first-in-first-out buffers).
\item From the \Aa description, a virtual
circuit (described in a virtual circuit
description language \vC) is generated.
The chief optimizations carried out at this
step are dependency based operation
ordering, dynamic loop-pipelining and decomposition of the system memory
into disjoint spaces based on static pointer
analysis (this considerably improves the
available memory bandwidth and reduces
system cost).  A \vC description also
consists of modules: however, the modules
are presented in a factored form (control X data X storage).
\item From the \vC description, a
VHDL description of the system is generated.
The system consists of modules, memory spaces
and FIFO buffers.  The modules are further
broken down into a control-path (a live and safe
Petri net), and a data-path (a graph of operators
and wires).
The chief optimization carried out at this
stage is resource sharing.  The \vC description
is analyzed to identify operations which
cannot be concurrently active and this information
is used to reduce the hardware required.
\item 
The VHDL description produced from 
\vC is in terms of a library
of VHDL design units which has been developed
as part of the AhirV2  effort.  This library
consists of control-flow elements, data-path
elements and memory elements.
\end{itemize}

Thus, to generate hardware using the AhirV2 flow,
it is possible to start at the C-level, at the \Aa level
at the \vC level or at the VHDL level (or a combination
of all these levels).  Starting at a higher level is
easier for the programmer, but using lower level representations
will usually lead to more efficient hardware.
Typically, if we start from C, circuits
produced by the AhirV2 flow are upto two orders of
magnitude more energy-efficient than
a processor \cite{ref:dsd2010}.  

Currently, there are only two restrictions in mapping
a C program to VHDL using the AhirV2 flow:
\begin{itemize}
\item No recursion, no cycles in the call-graph of the original
program.
\item No function pointers.
\end{itemize}

\section{Use Scenario 1:  a transformational system}

Many programs are transformational in nature.  There
is some input data $X$, and the program can be modeled as
a function $f$ which acts on $X$ to produce output data $Y$.

\subsection{A trivial example} \label{sec:Example}


Consider the following trivial example:
\begin{verbatim}
int add(int a, int b)
{
        int c = (a+b);
        return(c);
}
\end{verbatim}
We wish to generate a circuit which {\em implements}
the specification implied by this program.

We convert the program to LLVM byte code using
the {\bf clang} compiler (www.llvm.org)
\begin{verbatim}
      clang -std=gnu89 -emit-llvm -c add.c
\end{verbatim}
This produces a binary file {\bf add.o} which is
the LLVM byte-code.  To make the byte-code human
readable, we dis-assemble it using an LLVM utility
\begin{verbatim}
     llvm-dis add.o
\end{verbatim}
This is what the LLVM assembly code looks like
\begin{verbatim}
; ModuleID = 'add.o'
target datalayout = "e-p ...... "
target triple = "i386-pc-linux-gnu"

define i32 @add(i32 %a, i32 %b) nounwind {
  %1 = alloca i32, align 4
  %2 = alloca i32, align 4
  %c = alloca i32, align 4
  store i32 %a, i32* %1, align 4
  store i32 %b, i32* %2, align 4
  %3 = load i32* %1, align 4
  %4 = load i32* %2, align 4
  %5 = add nsw i32 %3, %4
  store i32 %5, i32* %c, align 4
  %6 = load i32* %c, align 4
  ret i32 %6
}
\end{verbatim}
To get to this point, we could have used several
optimizations which are available in the LLVM frame-work.
But we work with the unoptimized version to illustrate
the storage decomposition which is carried out by
the AhirV2 tools.

The LLVM byte-code is our starting point.  We first convert it
to \Aa.
\begin{verbatim}
llvm2aa add.o | vcFormat > add.o.aa
\end{verbatim}
This produces an \Aa program
\begin{verbatim}
// Aa code produced by llvm2aa (version 1.0)
$module [add]
// arguments
$in (a : $uint<32> b : $uint<32> )
$out (ret_val__ : $uint<32>)
$is
{
  $storage stored_ret_val__ : $uint<32>
  $branchblock [add]
  {
    //begin: basic-block bb_0
    $storage iNsTr_0 : $uint<32>
    $storage iNsTr_1 : $uint<32>
    $storage c : $uint<32>
    iNsTr_0 := a
    iNsTr_1 := b
    // load
    iNsTr_4 := iNsTr_0
    // load
    iNsTr_5 := iNsTr_1
    iNsTr_6 := (iNsTr_4 + iNsTr_5)
    c := iNsTr_6
    // load
    iNsTr_8 := c
    stored_ret_val__ := iNsTr_8
    $place [return__]
    $merge return__ $endmerge
    ret_val__ := stored_ret_val__
  }
}
\end{verbatim}

Now, this \Aa code is converted to a virtual circuit \vC representation.
\begin{verbatim}
     Aa2VC -O add.o.aa | vcFormat > add.o.aa.vc
\end{verbatim}
The virtual circuit representation is a bit too verbose to reproduce entirely
here, but we show some critical fragments
\begin{verbatim}
$module [add] 
{
  $in a:$int<32> b:$int<32>
  $out ret_val__:$int<32>
  $memoryspace [memory_space_0] 
  {
    $capacity 1
    $datawidth 32
    $addrwidth 1
    // ret-val is kept here
    $object [xxaddxxstored_ret_val__] : $int<32>
  }
  $memoryspace [memory_space_1] 
  {
    $capacity 1
    $datawidth 32
    $addrwidth 1
    // a is kept here.
    $object [xxaddxxaddxxiNsTr_0] : $int<32>
  }
  $memoryspace [memory_space_2] 
  {
    $capacity 1
    $datawidth 32
    $addrwidth 1
    // b is kept her
    $object [xxaddxxaddxxiNsTr_1] : $int<32>
  }
  $memoryspace [memory_space_3] 
  {
    $capacity 1
    $datawidth 32
    $addrwidth 1
    // c is kept here.
    $object [xxaddxxaddxxc] : $int<32>
  }
  $CP 
  {
     // a control-flow petri-net..  verbose..
  }
  // end control-path
  $DP 
  {
     // wires and operators.
  }

   // links between CP and DP
}
\end{verbatim}

Note that the stored objects a,b,c and
ret\_val\_\_ are mapped to different memory spaces.  Thus, the
chief difference between a \vC description and a processor is 
that the \vC program partitions storage into small units which
are accessed only by operators that need them. 

Finally, we take the \vC description and convert it to
VHDL
\begin{verbatim}
vc2vhdl -t add -f add.o.aa.vc | vhdlFormat > system.vhdl
\end{verbatim}
This produces a VHDL implementation of the system with
{\bf add} marked as a top-level module.  The VHDL that
is produced is too voluminous to reproduce here, but
the top-level system entity is
\begin{verbatim}
entity test_system is  -- system
  port (--
    add_a : in  std_logic_vector(31 downto 0);
    add_b : in  std_logic_vector(31 downto 0);
    add_ret_val_x_x : out  std_logic_vector(31 downto 0);
    add_tag_in: in std_logic_vector(0 downto 0);
    add_tag_out: out std_logic_vector(0 downto 0);
    add_start_req : in std_logic;
    add_start_ack : out std_logic;
    add_fin_req   : in std_logic;
    add_fin_ack   : out std_logic;
    clk : in std_logic;
    reset : in std_logic); --
  --
end entity;
\end{verbatim}

The AHIR system is a purely synchronous implementation
which uses only the rising edge of the {\bf clk} input.
The {\bf reset} is synchronous, and is active high. 

The VHDL implementation of a module in the AHIR system
corresponds to that of a pipeline stage (see Figure \ref{fig:ModulePipeline}).
\begin{figure}
\begin{centering}
\centerline{\psfig{figure=ModulePipeline.eps,width=3.0in,height=3.0in}}
\caption{System and Module Interfaces}
 \label{fig:ModulePipeline}
\end{centering}
\end{figure}
The ports 
\begin{verbatim}
add_a add_b
\end{verbatim}
correspond to the input arguments of the top-level 
function {\bf add}.  The port 
\begin{verbatim}
ret_val_x_x
\end{verbatim}
corresponds to the value returned by {\bf add}.
The pair of ports
\begin{verbatim}
add_start_req  add_start_ack
\end{verbatim}
implement a {\bf start} protocol; The environment
asserts {add\_start\_req} when it wants to start
{\bf add}, and the AHIR system asserts {\bf add\_start\_ack}
whenever it is ready to start.  The environment is required
to hold the input arguments steady until it observes
the acknowledge from the AHIR system.
The pair of ports
\begin{verbatim}
add_fin_req  add_fin_ack
\end{verbatim}
implement a {\bf finish} protocol; The environment
asserts {add\_fin\_req} when it is in a position to accept
the returned value from a previously started 
{\bf add}, and the AHIR system asserts {\bf add\_fin\_ack}
whenever it has a return-value avaiable.  The returned value
is valid only when the acknowledge from the AHIR system is
asserted. 
The ports
\begin{verbatim}
    add_tag_in  add_tag_out
\end{verbatim}
provide a mechanism by which a tag can be presented
by the environment to the AHIR system to identify
a particular invocation of the {\bf add} function.  The
width of the tag is chosen (by the AhirV2 tool-chain,
specifically, the vc2vhdl tool) to be large enough
that each active call to {\bf add} can be identified
uniquely.


\section{Use Scenario 2: pipelines}

A pipeline is a collection of parallel processes which
work together to accomplish a certain function (or to finish
a job).   Pipelines are commonly used in software and
in hardware systems.

The concept of pipes or sockets provide a natural mechanism
for communication between parallel processes in a software
pipeline.  In \Aa (and \vC), pipes are first-in-first-out
buffers which provide a corresponding communication construct
in the generated hardware.

\subsection{An example of a trivial pipeline}


Consider the following pipeline which has two stages
{\bf foo} and {\bf bar}:  
\begin{verbatim}
Environment --> foo --> bar --> Environment
\end{verbatim}
The stage {\bf foo} takes a 32-bit
integer from the external world, complements it and passes it on to stage
bar.  The stage {\bf bar} takes the 32-bit number from {\bf foo}, complements
it and sends it to the external world (in effect, nothing useful
is done, this is only an illustration).

If we are writing this pipe-line as a program, we could implement
two independent processes (or threads) and use named pipes to
perform the communication between these processes/threads (and the
``outside´´ world).  For example:
\begin{verbatim}
#include <iolib.h>
void foo()
{
   while(1)
   {
     uint32_t data = read_uint32("inpipe");
     write_uint32("midpipe", ~data);
   }
}

void bar()
{
   while(1)
   {
     uint32_t data = read_uint32("midpipe");
     write_uint32("outpipe", ~data);
   }
}
\end{verbatim}
In this example, the read/write functions are provided
as part of a {\em pipeHandler} library which is bundled with
the AhirV2 distribution.

The two functions can be compiled into separate processes
or can be used in threads in a multi-threaded program (using pthreads, for
example), and one gets a software pipeline, in which
the inter-process communication is done using the
read/write function calls, which use named pipes.

One can also map this pipeline to a VHDL system, using the
following flow.  One uses the standard flow that has already
been described:
\begin{verbatim}
clang -std=gnu89 -I../../../iolib/ -emit-llvm -c prog.c
llvm2aa prog.o | vcFormat >  prog.o.aa
Aa2VC -O -I mempool -C prog.o.aa | vcFormat > prog.o.aa.vc
vc2vhdl -C -s ghdl -T foo -T bar -f prog.o.aa.vc\
           | vhdlFormat > system.vhdl
\end{verbatim}
Note that in this case, we use the -T option in vc2vhdl to
specify that foo and bar are free-running top-level modules.
\begin{figure}
\begin{centering}
\centerline{\psfig{figure=FooBarPipeline.eps,width=3.0in,height=3.0in}}
\caption{Hardware implementation of Foo-Bar pipeline}
 \label{fig:FooBarPipeline}
\end{centering}
\end{figure}
The resulting AHIR system in system.vhdl implements the pipeline
(the block diagram of the system is shown in Figure \ref{fig:FooBarPipeline})
with the following interface:
\begin{verbatim}
entity test_system is  -- system 
  port (-- 
    clk : in std_logic;
    reset : in std_logic;
    inpipe_pipe_write_data: in std_logic_vector(31 downto 0);
    inpipe_pipe_write_req : in std_logic_vector(0 downto 0);
    inpipe_pipe_write_ack : out std_logic_vector(0 downto 0);
    outpipe_pipe_read_data: out std_logic_vector(31 downto 0);
    outpipe_pipe_read_req : in std_logic_vector(0 downto 0);
    outpipe_pipe_read_ack : out std_logic_vector(0 downto 0)); -- 
  -- 
end entity;
\end{verbatim}
The system has interfaces corresponding to the pipes inpipe and
outpipe through which data is exchanged.

In practice, one can have any number of pipes and 
interacting processes in a pipeline implementation.


\section{Storage variables and memory spaces}


In an \Aa program, variables can be of three kinds:
storage variables, pipe variables, or single-static-assignment
variables.  Storage variables are implemented in memory,
pipe variables as FIFO buffers, and single-static-assignment
variables as registers.  

While transforming an \Aa description down to a \vC description,
the storage variables in the \Aa program need to be grouped
into memory spaces.  Two storage variables are put in the 
same memory space only if we
determine that a pointer de-reference in the \Aa
program can point into either of the two storage variables
(this is determined by a conservative static analysis).

A memory space in a \vC description is characterized by
a word-length (the greatest common divisor of the widths
of accesses to this memory space), an address-width (wide enough
to allow access to all words in the memory space), and
a capacity (the number of words in the memory space).  Typically,
a program will have many small memory spaces corresponding to
scratch storage and some large memory spaces which correspond
to arrays etc.


There is one small issue, however.
Consider the following C program:
\begin{verbatim}
int main(int* b)
{
   int q[2];
   q[0] = *b;
   q[1] = q[0];
   return(q[1]);
}
\end{verbatim}
When this program is mapped to a circuit, we identify 
two distinct memory spaces, one which contains the
array $q$ and the other corresponding to the 
external world (the one referred to by the pointer $b$).
Where is the external memory physically located?  In 
the AhirV2 flow, we can either locate it outside
the system or inside the system which is being 
described by this program.

If the external memory is to be placed outside, then
accesses to it from within the system must be routed
outside the system.  On the other hand if it is
to be placed inside, a storage object corresponding 
to it must be created and all accesses to the external
memory must be directed at this storage object.
Further, the external world must have a mechanism for
accessing this storage object.

Both options are supported in the AhirV2 flow through
{\bf AaLinkExtMem}.


\subsection{Keeping the external memory outside the system}

In this scenario, all memory accesses which are resolved
to be to a storage object which is not declared in the
\Aa program are redirected outside the system by using
pipes.

If you want to keep the external memory
outside, you will have to go through the following sequence
\begin{verbatim}
# first use clang (or llvm-gcc) to generate llvm-byte-code
clang -std=gnu89 -emit-llvm -c foo.c
#
# disassemble so that you can make sense of the llvm bc.
llvm-dis foo.o
#
# OK, now take the llvm byte code
# and generate an Aa description.
# use the storageinit option to initialize
# global storage.
# (the pipe to vcFormat is to prettify the output)
 llvm2aa -storageinit foo.o | vcFormat > foo.o.aa
#
#
# Do an Aa -> Aa transformation: map external
# memory outside..
AaLinkExtMem foo.o.aa | vcFormat > foo.o.memlinked.ExternalOutside.aa
#
# Now take the Aa code and generate a virtual
# circuit..
# the -O flag does dependency analysis in straight-line
# code and parallelizes it.
#
Aa2VC -O foo.o.memlinked.ExternalOutside.aa | vcFormat\
                 > foo.o.memlinked.ExternalOutside.aa.vc
#
# finally, generate vhdl from the vc description.  Note that
# you will have to mark the module foo as well as the
# extmem_store_32/load_32 modules as top-level modules
# so that it is possible for the outside world to serve
# requests made from inside.
#
vc2vhdl -O -t foo -t extmem_store_32 -t extmem_load_32\ 
         -f foo.o.memlinked.ExternalOutside.aa.vc | vhdlFormat\
              > foo_o_aa_memlinked_external_outside_vc.vhdl

\end{verbatim}

If you look at the generated top-level VHDL entity, its ports
will be
\begin{verbatim}
entity test_system is  -- system
  port (--
    clk : in std_logic;
    reset : in std_logic;

    --   some-lines-omitted --
    --   foo-related ports --
    --   some-lines-omitted --

    extmem_read_address_32_pipe_read_data: out std_logic_vector(31 downto 0);
    extmem_read_address_32_pipe_read_req : in std_logic_vector(0 downto 0);
    extmem_read_address_32_pipe_read_ack : out std_logic_vector(0 downto 0);
    extmem_read_data_32_pipe_write_data: in std_logic_vector(31 downto 0);
    extmem_read_data_32_pipe_write_req : in std_logic_vector(0 downto 0);
    extmem_read_data_32_pipe_write_ack : out std_logic_vector(0 downto 0);
    extmem_write_address_32_pipe_read_data: out std_logic_vector(31 downto 0);
    extmem_write_address_32_pipe_read_req : in std_logic_vector(0 downto 0);
    extmem_write_address_32_pipe_read_ack : out std_logic_vector(0 downto 0);
    extmem_write_data_32_pipe_read_data: out std_logic_vector(31 downto 0);
    extmem_write_data_32_pipe_read_req : in std_logic_vector(0 downto 0);
    extmem_write_data_32_pipe_read_ack : out std_logic_vector(0 downto 0)); --
  --
end entity;
\end{verbatim}
The external memory read and write address and data are clearly
visible.   The outside world is responsible for serving the
read/write requests made from the inside.

\subsection{Keeping the external memory inside the system}

In this scenario, we will assume that all accesses to 
storage variables not defined in the \Aa program are 
to be directed to a storage variable which is to be
a viewed as a shared memory pool that is visible to the
\Aa program as well as to the outside world.  The visibility
to the outside world is provided by access functions
that the outside world can use to read/write from this
shared memory pool.

You will have to go through the following sequence:
\begin{verbatim}
# use clang (or llvm-gcc) to generate llvm-byte-code
clang -std=gnu89 -emit-llvm -c foo.c
#
# disassemble so that you can make sense of the llvm bc.
llvm-dis foo.o
#
# OK, now take the llvm byte code
# and generate an Aa description.
# use the storageinit option to initialize
# global storage.
# (the pipe to vcFormat is to prettify the output)
 llvm2aa -storageinit foo.o | vcFormat > foo.o.aa
#
#
# Do an Aa -> Aa transformation: map external
# memory to a storage area inside the system...
# -I 1024 says that the amount of memory that will be
# referred to is 1024 bytes.
# -E mempool says that the storage object corresponding
# to external memory is named mempool.
AaLinkExtMem  -I 1024 -E mempool foo.o.aa | vcFormat\
               > foo.o.memlinked.ExternalInside.aa
#
# Now take the Aa code and generate a virtual
# circuit..
# the -O flag does dependency analysis in straight-line
# code and parallelizes it.
# the -I mempool option says that external memory is
# to be mapped inside the system to object mempool..
#
Aa2VC -O -I mempool foo.o.memlinked.ExternalInside.aa\
           | vcFormat > foo.o.memlinked.ExternalInside.aa.vc
#
# finally, generate vhdl from the vc description.
# note that you will have to mark mem_load__ and mem_store__
# as top-level modules, so that the external world can
# access its memory pool inside the system.
#
vc2vhdl -O -t foo -t mem_load__ -t mem_store__ \ 
     -f foo.o.memlinked.ExternalInside.aa.vc\
       | vhdlFormat > foo_o_aa_memlinked_external_inside_vc.vhdl
\end{verbatim}

In this example, we are saying that the shared memory pool
variable is named mempool, and it is an array of 1024 {\bf bytes}.
The generated top-level VHDL entity has the following
ports:
\begin{verbatim}
entity test_system is  -- system
  port (--
    foo_b : in  std_logic_vector(31 downto 0);
    foo_ret_val_x_x : out  std_logic_vector(31 downto 0);
    foo_tag_in: in std_logic_vector(0 downto 0);
    foo_tag_out: out std_logic_vector(0 downto 0);
    foo_start : in std_logic;
    foo_fin   : out std_logic;
    mem_load_x_x_address : in  std_logic_vector(31 downto 0);
    mem_load_x_x_data : out  std_logic_vector(7 downto 0);
    mem_load_x_x_tag_in: in std_logic_vector(0 downto 0);
    mem_load_x_x_tag_out: out std_logic_vector(0 downto 0);
    mem_load_x_x_start_req : in std_logic;
    mem_load_x_x_start_ack : out std_logic;
    mem_load_x_x_fin_req   : in std_logic;
    mem_load_x_x_fin_ack   : out std_logic;
    mem_store_x_x_address : in  std_logic_vector(31 downto 0);
    mem_store_x_x_data : in  std_logic_vector(7 downto 0);
    mem_store_x_x_tag_in: in std_logic_vector(0 downto 0);
    mem_store_x_x_tag_out: out std_logic_vector(0 downto 0);
    mem_store_x_x_start_req : in std_logic;
    mem_store_x_x_start_ack : out std_logic;
    mem_store_x_x_fin_req   : in std_logic;
    mem_store_x_x_fin_ack   : out std_logic;
    clk : in std_logic;
    reset : in std_logic); --
  --
end entity;
\end{verbatim}

The system provides  memory load and memory store
function interfaces to the external world 
(through  mem\_load.. and mem\_store..).
The shared memory variable is guaranteed to have a base address of $0$.  Thus,
byte mempool[I] will be present at address $I$.

\section{The tools}

We assume that you have access to either {\bf llvm-gcc}
or {\bf clang} as the front-end compiler which generates
LLVM byte-code from C/C++.  The current AhirV2 toolset
is consistent with llvm 2.8 and clang 2.8.

The other tools in the chain are described below.

\subsection{{\bf llvm2aa}}

This tool takes LLVM byte code and converts it into an
\Aa file.
\begin{verbatim}
llvm2aa options bytecode.o > bytecode.aa
\end{verbatim}
The generated \Aa code is sent to {\bf stdout} and all informational
messages are sent to {\bf stderr}.  On success, the tool returns 0.

The options:
\begin{itemize}
\item {\bf -modules=listfile} : Specify the list of functions in the bytecode
which should be converted to \Aa.   The names of these functions should be
listed in the text-file listfile. If absent, all functions
are converted.
\item {\bf -storageinit} :  Storage objects in the llvm bytecode
are explicitly initialized in the generated \Aa code.   An initializer
routine named
\begin{verbatim}
    global_storage_initializer
\end{verbatim}
is instantiated in
the \Aa code for this purpose.
\item {\bf -pipedepths=filename} : Specifies a file which contains
the depths of pipes which are part of the generated \Aa code.
\item {\bf -extract\_do\_while} : Innermost loops which are marked
using a call to the special function 
\begin{verbatim}
    _loop_pipelining_on_
\end{verbatim}
are extracted as pipelined do-while loops.  This is necessary
for automatic cross-iteration parallelization of inner loops in the 
generated hardware (substantial performance benefits can be
realized).
\end{itemize}

\subsection{{\bf AaLinkExtMem}}

This linker tool takes a list of \Aa files, elaborates the program,
creates a global storage initializer, and
does memory space decomposition.  The externally visible memory space is
linked in one of two ways: either it is assumed to be external
and all accesses to it are routed out of the \Aa program,
or it is assumed to be internal and assumed to consist of
a memory object (an array of bytes).  External pointer dereferences
are handled as if they are directed at this memory object.
\begin{verbatim}
AaLinkExtMem options file1.aa file2.aa ...  > linked.aa
\end{verbatim}
The generated \Aa code is sent to {\bf stdout} and all informational
messages are sent to {\bf stderr}.  On success, the tool returns 0.

The options:
\begin{itemize}
\item {\bf -I n}: specifies that external references to memory
are to be mapped as if they are to an internal object whose size
is $n$ bytes.
\item {\bf -E obj-name} : specifies that the object to which
external references are mapped is to be named obj-name.
\end{itemize}
We recommend that you use the {\bf -I} and {\bf -E} options to
locate externally visible memory into a specified object in the
\Aa program.   

If the {\bf -I} option is not used, then all external memory
references are routed out of the \Aa program through pipes.
In this case, if the \Aa compiler determines that there is some pointer 
in the program which can point
to both internal and external memory, then this will be
declared as an error!  

If the programs being linked contain memory initialization
routines, the linker generates a global storage initialization
function which is named {\bf global\_storage\_initializer\_}.
This global initializer calls all the memory initializers in
the programs being linked.

\subsection{{\bf AaOpt}}

The optimization utility {\bf AaOpt} takes an \Aa program (list of
\Aa files) and produces an optimized version of the source program.
\begin{verbatim}
AaOpt options file1.aa file2.aa ... > optimized.aa
\end{verbatim}
The optimized \Aa code is printed to {\bf stdout}.  On success, 
the tool returns a 0 (else a non-zero).  Macro and inlined
function calls in the source code are substituted in place
in the optimized code.

The options:
\begin{itemize}
\item {\bf -r module-name} (optional) : specifies a root module in the
system.  Multiple root modules can be specified.  All dead code (which
is not reachable from a root module) is eliminated.
\item {\bf -I extmem-object} (optional) : similar to AaLinkExtMem,
this option specifies the name of the extmem-object in the
source \Aa files.
\item {-B} (optional) : if specified, add buffering to balance pipelined
loops so that loop performance is not bottlenecked by inadequate
buffering.
\end{itemize}


\subsection{{\bf Aa2VC}}

This tool takes a list of \Aa programs and converts them
to a \vC description. 
\begin{verbatim}
Aa2VC options file1.aa file2.aa ...  > result.vc
\end{verbatim}
The generated \vC code is sent to {\bf stdout} and all informational
messages are sent to {\bf stderr}.  On success the tool returns 0.

The options:
\begin{itemize}
\item {\bf -O} : if used, sequential statement blocks are parallelized
by doing dependency analysis.
\item {\bf -C} : if used, a C stub is created for every module that
is not called from within the system.  These stubs can be used to
interface to a VHDL simulator (or even drive hardware) to verify
the VHDL code generated by downstream tools.
\item {\bf -U} : memory subsystems will be unordered (that is,
will not guarantee in-order completion of accesses).  This
leads to a simpler memory subsystem, but more conservative
control flow.  The default is that all memory subsystems
are ordered (will complete read/write requests in the order that
they are accepted).
\item {\bf -r root-module} (optional): specifies a root module.
Code which is not accessible from a root-module is considered
as dead code and is ignored.
\item {\bf -I obj-name} : if specified, all external memory references
are considered as being directed at the storage object named obj-name.
If not specified, then the tool will throw an error if it finds
a pointer dereference that cannot be resolved as pointing only to
storage objects declared inside the \Aa program.
\end{itemize}

\subsection{\bf vc2vhdl}

Takes a collection of \vC descriptions and converts them to
an AHIR system described in VHDL.
\begin{verbatim}
vc2vhdl [-O] [-C] [-q] [-a] [-e <entity-name] [-w]\
         -t/-T foo [-t/-T bar -t/-T bar2 ...]\
         -f file1.vc -f file2.vc ...  > system.vhdl
\end{verbatim}

The options:
\begin{itemize}
\item {\bf -t} : to specify the modules which are to be 
accessible from the ports of the generated VHDL system.
Such modules have to be top-level (that is, they cannot
be called from within the program).
Multiple top-level modules can be specified in this way.
The control and argument ports for these modules are
visible at the interface of the generated AHIR system.
\item {\bf -T} : to specify top-level modules which are to be 
free-running inside the AHIR system.
Multiple top-level modules can be specified in this way.
Such modules do not have any arguments and do not return
any values.  Their only mechanism of communication with
the world outside the AHIR system is through pipes.
The control ports for these modules are
{\bf not visible} at the interface of the generated AHIR system.
In the AHIR system, these modules are started on reset
and are run forever (restarted after they finish, forever).
\item {\bf -f file-name} : specifies the \vC files to be analyzed. 
Multiple \vC files may be specified.  An object must be
defined before it is used, so the \vC files must be 
specified in the correct order.
\item {\bf -O} : optimize the generated VHDL by compacting
the control-path.  This does not change the resulting
hardware, but makes the generated VHDL file smaller.
\item {\bf -C} : the VHDL code has a system test bench which
interfaces to foreign code using a VHPI/Modelsim-FLI interface.
If this is not specified, the  generated test bench simply
instantiates the system and starts all top-level modules
off (you will need to fill in your own test bench here).
The C testbench is usually easier to write (it probably
already exists in the form of the original program).
\item {\bf -a} : try to minimize the area of the resulting
VHDL by sharing operators to the maximum extent possible
(allowing potential contention for resources).  This will
result in a slower (usually by 2X) system, but will
also reduce the area (usually by 0.5X).  If not specified,
two operations will be mapped to the same
operator  only if it can be proved that they cannot be active simultaneously.
\item {\bf -q } : if specified, do aggressive register insertion
to minimize the clock period.
\item {\bf -S bypass-stride } : by specifying the bypass stride
(an integer $\geq 1$),
the user can trade-off clock cycles versus clock period.  
The lowest clock period will be obtained for -S 1.   
\item {\bf -e top-entity-name} : The generated top-level VHDL entity
corresponding to the AHIR system is named top-entity-name.  The default
is test\_system.
\item {-L function-library} : AhirV2 provides some built in operator
functions which can be called from your code.  These are organized
as function libraries and this option specifies a function library
to look into when generating VHDL.  For example {\em -L fpu} gives
access to the floating point library which provides some useful
built in functions (e.g. fpalu32, fpalu64 etc.).
\item {\bf -w} :  If specified, the VHDL system and test-bench are
generated as separate unformatted VHDL files.  You will need to
format these using the vhdlFormat command.
\item {\bf -s ghdl/modelsim} :  If {\bf ghdl} is specified
with the -s option, then the generated testbench (if -C is specified)
uses the VHPI interface to link with foreign code.  Otherwise,
the generated testbench (if -C is specified) uses the Modelsim FLI
interface to link with foreign code.
\end{itemize}

The tool performs concurrency analysis to determine operations which
can be mapped to the same physical operator without the need for
arbitration.  It also instantiates separate memory subsystems for
the disjoint memory spaces (in practice many of the memory spaces
are small and are converted to register banks).


\subsection{{\bf Aa2C}: convert an {\bf Aa} description into a {\bf C} program}

The AhirV2 flow offers considerable flexibility to a system designer.
For example, it is possible to write code directly in {\bf Aa} in order
to get more optimal implementations (relative to those obtained 
starting from {\bf C}).  In such cases, if we wish to simulate
the {\bf Aa} description, we would use the {\bf Aa2C} utility to
convert the {\bf Aa} code to ANSI {\bf C}, and then compile it
in the usual way.

The {\bf Aa2C} program can be summarized as
\begin{verbatim}
Aa2C [-I <ext-mem-object>] <aa-file> (<aa-file>)*
\end{verbatim}
The only option is:
\begin{itemize}
\item {\bf -I <ext-mem-object>}:  the same behaviour as in {\bf Aa2VC}.
\end{itemize}
The remaining arguments are {\bf Aa} files which will be linked and
converted to {\bf C} code.  Two outputs files are created:
\begin{itemize}
\item aa\_c\_model.h :  a header file declaring functions in the generated
source code.
\item aa\_c\_model.c : a source file containing function definitions
corresponding to the {\bf Aa} modules.
\end{itemize}

External calls into the generated {\bf C} code must have the
form:
\begin{verbatim}
void foo ( Ctype_1 in_1, Ctype_2 in_2, Ctype_3* out_1, Ctype_4* out_2);
\end{verbatim}
where Ctype is either a float or double or (int/uint)(64/32/16/8)\_t type.
You can then link your external code with the generated {\bf C} code
in the usual way.

\subsubsection{Restrictions in using {\bf Aa2C}}

The current implementation of {\bf Aa2C} produces un-threaded code.
Thus, if you have a parallel block in your {\bf Aa} code, the
statements in the parallel block are serialized in the resulting
{\bf C} program.  This can result in the generated {\bf C} program
potentially hanging (if one of the statements in the parallel
block runs for-ever).   Another situation is when two
concurrent blocks in the {\bf Aa} program are writing and reading
from the same pipe.  In such a case, the serialized code may
get dead-locked.  You need to be careful that your {\bf Aa} code
does not have such situations (the simplest option is
to not use parallel blocks in the {\bf Aa} code!).

This issue will be fixed in a future release of {\bf Aa2C}.

\subsection{Miscellaneous: {\bf vcFormat} and {\bf vhdlFormat}}

The outputs produced by {\bf Aa2VC} and {\bf vc2vhdl} are
not well formatted.  One can format \Aa and \vC files
using  {\bf vcFormat} as follows
\begin{verbatim}
vcFormat < unformatted-vc/aa-file  > formatted-vc/aa-file
\end{verbatim}
and similarly use {\bf vhdlFormat} to format generated
VHDL files.


\begin{thebibliography}{99}
\bibitem{ref:dsd2010}
Sameer D. Sahasrabuddhe, Sreenivas Subramanian, Kunal P. Ghosh, Kavi Arya, Madhav P. Desai, 
"A C-to-RTL Flow as an Energy Efficient Alternative to Embedded Processors in Digital Systems," 
DSD, pp.147-154, 2010 13th Euromicro Conference on Digital 
System Design: Architectures, Methods and Tools, 2010
\end{thebibliography}
\end{document}
