% Created 2010-02-14 Sun 14:27
\documentclass[12pt,a4paper]{article}
\usepackage{graphicx}
\usepackage{times}
\usepackage[margin=0.75in]{geometry}
\newtheorem{lemma}{Lemma}
\newtheorem{definition}{Definition}

\title{Testing for compatibility in the LRG}
%\author{Sameer D. Sahasrabuddhe}
\date{[2010-02-14 Sun]}

\begin{document}

\maketitle


\section{Algorithm}
\label{sec-1}


Given two vertices $u$ and $v$ in an LRG $G$, we test whether they are
compatible using the following algorithm.

\begin{enumerate}
\item Traverse depth-first starting from the vertex $u$, towards the
    root. Mark every vertex $x$ encountered in the traversal and
    record the fork indicated by the edges along which we reached the
    vertex $x$. Every such edge either indicates the same fork, or
    there is only one unlabelled edge along which $x$ is reachable as
    shown in Lemma \ref{all-or-none}.
\item Now traverse the LRG depth-first starting from $v$. When a marked
    vertex $x$ is reached then:

\begin{enumerate}
\item Return true (the two nodes are compatible) in each of the
       following cases:

\begin{enumerate}
\item The edge along which we reached $x$ is not labelled.
\item No fork is recorded at $x$.
\item The edge indicates a fork different from the one recorded at
       $x$.
\end{enumerate}

\item If none of the above three cases holds, then retrace the edge
       along which we reached $x$, and continue the depth-first
       traversal. The edges reachable beyond $x$ are not examined
       since they do not provide any conclusive information, as shown
       in Lemma\ref{look-no-further}.
\end{enumerate}

\end{enumerate}
\section{Details}
\label{sec-2}


Consider a Type-2 petri-net $P$ and its associated LRG $G$. Let $F$ be
a canonical fork region in $P$, such that all the child Class $C$
regions are replaced by simple places. $G'$ is the subgraph induced on
LRG $G$ by the labels assigned to the forks, joins and simple places
in the region $G$.

\begin{definition}

A transition $t_1\in F$ is said to be connected to a transition
$t_2\in F$ {\bf through a simple place} $p \in F$ when there exist
edges $(t_1,p),(p,t_2) \in F$.

\end{definition}

\begin{lemma}

Two transitions $t_1, t_2 \in F$ have the same label $v \in G'$ iff
one element is a join (say $t_1$) and the other element ($t_2$) such
that $t_1$ is connected to $t_2$ through a simple place $p$.
Consequently, the place $p$ is the only place in $F$ with label $v$.

\end{lemma}

\paragraph{Proof:}

\begin{description}
\item [Statement 1] Join $t_1$ is connected to fork $t_2$ through a
   simple place $p$.

   From the way labels are constructed in a Type-2 petri-net, the
   label assigned to the place $p$ and the fork $t_2$ is the same as
   the label assigned to join $t_1$.
\item [Statement 2] Transitions $t_1$ and $t_2$ have the same label.

   Consider transition $t_1\in F$ assigned a label $v_1\in G$. $t_1$
   is either the entry $f'$ of the fork region, or its label is
   derived from the label $v'$ assigned to $f'$. Every transition on
   the path from $f'$ to $t_1$ is either a fork that extends the
   label, or a join that creates a union with some other label. Let
   $S=[t_1,t_2,\ldots]$ be the sequence of forks and joins that occur
   along the path from $f$ to $t_1$.

   If transition $t_2$ has the same label $v_1$, then the path from
   $f$ to $t_2$ must pass through the same sequence $S$. At the end of
   this sequence, let $t_1$ occur first on the path. Since no
   modification occurs to the label when $t_2$ is encountered, the
   following statements must be true:

\begin{enumerate}
\item $t_1$ is not a fork. If it were a fork, it would extend the
      label on every outgoing edge. Hence, $t_1$ is a join.
\item There is no intervening fork or join on the path from $t_1$ to
      $t_2$, since this would also change the label that reaches
      $t_2$. Hence $t_1$ is connected to $t_2$ only through a simple
      place.
\item $t_2$ is not a join. It it were a join, it would generate a new
      label that is a union of the label at $p$ with other labels
      assigned to its other predecessors. Hence $t_2$ is a fork.
\end{enumerate}

\end{description}
\begin{lemma}

If $y_2$ is a vertex in G' with
unlabelled out-edges to distinct vertices $z_1,z_2,\ldots$, then at
most one vertex $z_i$ has out-egdges in the subgraph G'.

\end{lemma}

For vertices $u$ and $v$ in the LRG, let
\mbox{$E(u,v)=\{e_1,e_2,\ldots\}$} be the set of distinct out-edges at
$u$ along which $v$ is reachable from $u$. The following properties
hold for the set $E(u,v)$:

\begin{lemma}
\label{all-or-none}

If an edge $e_{i_0} \in E(u,v)$ is unlabelled, then $E(u,v) =
\{e_{i_0}\}$, else every edge $e_i \in E$ indicates the same fork $f$.

\end{lemma}


\end{document}